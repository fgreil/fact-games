\documentclass[a4paper]{article}
\usepackage[margin=12mm]{geometry}
\usepackage{graphicx}
\usepackage{array}
\usepackage{lmodern}
\usepackage[T1]{fontenc}
\usepackage{microtype}
\setlength{\parindent}{0pt}

% Card dimensions (59 x 91 mm)
\newlength{\cardW}
\newlength{\cardH}
\setlength{\cardW}{59mm}
\setlength{\cardH}{91mm}

% Command to produce a card frame with 9 parameters
\newcommand{\SeaCard}[9]{%
\fbox{\begin{minipage}[t][\dimexpr\cardH-2\fboxsep-2\fboxrule\relax][t]{\dimexpr\cardW-2\fboxsep-2\fboxrule\relax}
\raggedright
\setlength{\baselineskip}{8pt}
% Header with decorative lines
\noindent\hfill{\tiny\rule{8mm}{0.4pt} \textbf{MEERESGEBIETE} \rule{8mm}{0.4pt}}\hfill\mbox{}\\[0.5ex]
\noindent\rule{\linewidth}{1pt}\\[0.5ex]
% Title (German name)
{\Large\textbf{#1}}\\[0.2ex]
% English subtitle
{\scriptsize\textit{#2}}\\[0.3ex]
\noindent\rule{30mm}{0.4pt}\\[1ex]
% Illustration placeholder
\noindent\fbox{\begin{minipage}[c][18mm]{\dimexpr\linewidth-2\fboxsep-2\fboxrule\relax}
\centering
\vfill
{\scriptsize [Illustration]}
\vfill
\end{minipage}}\\[1ex]
% Intro text
{\scriptsize #3}\\[1ex]
\noindent\rule{\linewidth}{0.4pt}\\[0.3ex]
% Statistics table - removed Volume column
{\scriptsize
\begin{tabular}{@{}l@{\hspace{1mm}}|@{\hspace{1mm}}l@{}}
\textbf{Fläche:} #4 & \textbf{Mitt.\ Tiefe:} #5 \\
\end{tabular}
}\\[-0.5ex]
\noindent\rule{\linewidth}{0.4pt}\\[0.3ex]
{\scriptsize
\begin{tabular}{@{}l@{\hspace{1mm}}|@{\hspace{1mm}}l@{}}
\textbf{Max. Tiefe:} #6 & \textbf{Temp.:} #7 \\
\end{tabular}
}\\[-0.5ex]
\noindent\rule{\linewidth}{0.4pt}\\[0.3ex]
{\scriptsize
\begin{tabular}{@{}l@{}}
\textbf{Salzgehalt:} #8 \\
\end{tabular}
}\\[-0.5ex]
\noindent\rule{\linewidth}{0.4pt}\\[0.3ex]
% Anrainer and Zuflüsse (combined in parameter #9)
{\scriptsize #9}
\end{minipage}}%
}

\begin{document}
\pagestyle{empty}

%----------------------------------
% CARD 1: North Pacific Ocean
%----------------------------------
\SeaCard
{Nordpazifik} % 1: German Title
{North Pacific Ocean} % 2: English subtitle
{Der North Pacific Ocean ist einer der fünf Weltmeere und bedeckt einen bedeutenden Teil der Erdoberfläche. Besondere Merkmale: Tiefster Punkt (Mariana Trench), Pazifischer Feuerring.} % 3: Intro
{82\,500\,000\,km²} % 4: Fläche
{4\,280\,m} % 5: Mitt. Tiefe
{10\,911\,m} % 6: Max Tiefe
{0--30\,°C} % 7: Temperatur
{32--35\,‰} % 8: Salzgehalt
{\textbf{Anrainer:} USA,  Canada,  Russia,  Japan,  China, u.a.\\[0.3ex]
\noindent\rule{\linewidth}{0.4pt}\\[0.3ex]
\textbf{Zuflüsse:} Yukon, Columbia, Sacramento, Amur} % 9: Anrainer + Zuflüsse

\vspace{5mm}


%----------------------------------
% CARD 2: South Pacific Ocean
%----------------------------------
\SeaCard
{Südpazifik} % 1: German Title
{South Pacific Ocean} % 2: English subtitle
{Der South Pacific Ocean ist einer der fünf Weltmeere und bedeckt einen bedeutenden Teil der Erdoberfläche. Besondere Merkmale: Inselstaaten, Korallenriffe.} % 3: Intro
{82\,500\,000\,km²} % 4: Fläche
{4\,000\,m} % 5: Mitt. Tiefe
{7\,434\,m} % 6: Max Tiefe
{10--30\,°C} % 7: Temperatur
{34--36\,‰} % 8: Salzgehalt
{\textbf{Anrainer:} Indonesia, Australia, New Zealand, Chile, Peru, Ecuador, and island nations\\[0.3ex]
\noindent\rule{\linewidth}{0.4pt}\\[0.3ex]
\textbf{Zuflüsse:} Fly, Sepik, Waiapu} % 9: Anrainer + Zuflüsse

\vspace{5mm}


%----------------------------------
% CARD 3: North Atlantic Ocean
%----------------------------------
\SeaCard
{Nordatlantik} % 1: German Title
{North Atlantic Ocean} % 2: English subtitle
{Der North Atlantic Ocean ist einer der fünf Weltmeere und bedeckt einen bedeutenden Teil der Erdoberfläche. Besondere Merkmale: Golfstrom, wichtige Schifffahrtsrouten.} % 3: Intro
{41\,490\,000\,km²} % 4: Fläche
{3\,300\,m} % 5: Mitt. Tiefe
{8\,486\,m} % 6: Max Tiefe
{-2--28\,°C} % 7: Temperatur
{34--37\,‰} % 8: Salzgehalt
{\textbf{Anrainer:} USA, Canada, Mexico, Europe, North Africa, Caribbean nations\\[0.3ex]
\noindent\rule{\linewidth}{0.4pt}\\[0.3ex]
\textbf{Zuflüsse:} St. Lawrence, Mississippi} % 9: Anrainer + Zuflüsse

\vspace{5mm}


%----------------------------------
% CARD 4: South Atlantic Ocean
%----------------------------------
\SeaCard
{Südatlantik} % 1: German Title
{South Atlantic Ocean} % 2: English subtitle
{Der South Atlantic Ocean ist einer der fünf Weltmeere und bedeckt einen bedeutenden Teil der Erdoberfläche. Besondere Merkmale: Mid-Atlantic Ridge.} % 3: Intro
{43\,510\,000\,km²} % 4: Fläche
{3\,900\,m} % 5: Mitt. Tiefe
{8\,264\,m} % 6: Max Tiefe
{0--28\,°C} % 7: Temperatur
{34--37\,‰} % 8: Salzgehalt
{\textbf{Anrainer:} Brazil, Argentina, Uruguay, South Africa, Namibia, Angola\\[0.3ex]
\noindent\rule{\linewidth}{0.4pt}\\[0.3ex]
\textbf{Zuflüsse:} Paraná, Orange, Congo} % 9: Anrainer + Zuflüsse

\vspace{5mm}


%----------------------------------
% CARD 5: Indian Ocean
%----------------------------------
\SeaCard
{Indischer Ozean} % 1: German Title
{Indian Ocean} % 2: English subtitle
{Der Indian Ocean ist einer der fünf Weltmeere und bedeckt einen bedeutenden Teil der Erdoberfläche. Besondere Merkmale: Wärmster Ozean, Monsuneinfluss.} % 3: Intro
{70\,560\,000\,km²} % 4: Fläche
{3\,741\,m} % 5: Mitt. Tiefe
{7\,258\,m} % 6: Max Tiefe
{20--30\,°C} % 7: Temperatur
{34--36\,‰} % 8: Salzgehalt
{\textbf{Anrainer:} India,  Pakistan,  Bangladesh,  Myanmar,  Thailand, u.a.\\[0.3ex]
\noindent\rule{\linewidth}{0.4pt}\\[0.3ex]
\textbf{Zuflüsse:} Ganges-Brahmaputra, Indus, Zambezi} % 9: Anrainer + Zuflüsse

\vspace{5mm}


%----------------------------------
% CARD 6: Southern Ocean
%----------------------------------
\SeaCard
{Südlicher Ozean} % 1: German Title
{Southern Ocean} % 2: English subtitle
{Der Southern Ocean ist einer der fünf Weltmeere und bedeckt einen bedeutenden Teil der Erdoberfläche. Besondere Merkmale: Umgibt die Antarktis, starke Strömungen.} % 3: Intro
{21\,960\,000\,km²} % 4: Fläche
{3\,270\,m} % 5: Mitt. Tiefe
{7\,235\,m} % 6: Max Tiefe
{-2--10\,°C} % 7: Temperatur
{33.5--34\,‰} % 8: Salzgehalt
{\textbf{Anrainer:} Antarctica\\[0.3ex]
\noindent\rule{\linewidth}{0.4pt}\\[0.3ex]
\textbf{Zuflüsse:} None (glacial meltwater)} % 9: Anrainer + Zuflüsse

\newpage


%----------------------------------
% CARD 7: Arctic Ocean
%----------------------------------
\SeaCard
{Arktischer Ozean} % 1: German Title
{Arctic Ocean} % 2: English subtitle
{Der Arctic Ocean ist einer der fünf Weltmeere und bedeckt einen bedeutenden Teil der Erdoberfläche. Besondere Merkmale: Kleinster Ozean, Eisbedeckung.} % 3: Intro
{14\,060\,000\,km²} % 4: Fläche
{1\,205\,m} % 5: Mitt. Tiefe
{5\,450\,m} % 6: Max Tiefe
{-2--5\,°C} % 7: Temperatur
{30--34\,‰} % 8: Salzgehalt
{\textbf{Anrainer:} Russia, Norway, Iceland, Greenland (Denmark), Canada, USA (Alaska)\\[0.3ex]
\noindent\rule{\linewidth}{0.4pt}\\[0.3ex]
\textbf{Zuflüsse:} Ob, Yenisei, Lena, Mackenzie} % 9: Anrainer + Zuflüsse

\vspace{5mm}


%----------------------------------
% CARD 8: Barents Sea
%----------------------------------
\SeaCard
{Barentssee} % 1: German Title
{Barents Sea} % 2: English subtitle
{Das Barents Sea ist ein Randmeer des Arctic mit charakteristischen regionalen Merkmalen. Besondere Merkmale: Reiche Fischgründe, Öl- und Gasvorkommen.} % 3: Intro
{1\,400\,000\,km²} % 4: Fläche
{230\,m} % 5: Mitt. Tiefe
{600\,m} % 6: Max Tiefe
{-1--12\,°C} % 7: Temperatur
{34--35\,‰} % 8: Salzgehalt
{\textbf{Anrainer:} Russia, Norway\\[0.3ex]
\noindent\rule{\linewidth}{0.4pt}\\[0.3ex]
\textbf{Zuflüsse:} Pechora} % 9: Anrainer + Zuflüsse

\vspace{5mm}


%----------------------------------
% CARD 9: Kara Sea
%----------------------------------
\SeaCard
{Karasee} % 1: German Title
{Kara Sea} % 2: English subtitle
{Das Kara Sea ist ein Randmeer des Arctic mit charakteristischen regionalen Merkmalen. Besondere Merkmale: Fast ganzjährig gefroren.} % 3: Intro
{880\,000\,km²} % 4: Fläche
{110\,m} % 5: Mitt. Tiefe
{620\,m} % 6: Max Tiefe
{-2--6\,°C} % 7: Temperatur
{10--34\,‰} % 8: Salzgehalt
{\textbf{Anrainer:} Russia\\[0.3ex]
\noindent\rule{\linewidth}{0.4pt}\\[0.3ex]
\textbf{Zuflüsse:} Ob, Yenisei} % 9: Anrainer + Zuflüsse

\vspace{5mm}


%----------------------------------
% CARD 10: Laptev Sea
%----------------------------------
\SeaCard
{Laptewsee} % 1: German Title
{Laptev Sea} % 2: English subtitle
{Das Laptev Sea ist ein Randmeer des Arctic mit charakteristischen regionalen Merkmalen. Besondere Merkmale: Methanemissionen, 9 Monate gefroren.} % 3: Intro
{662\,000\,km²} % 4: Fläche
{578\,m} % 5: Mitt. Tiefe
{3\,385\,m} % 6: Max Tiefe
{-1.8--8\,°C} % 7: Temperatur
{20--34\,‰} % 8: Salzgehalt
{\textbf{Anrainer:} Russia\\[0.3ex]
\noindent\rule{\linewidth}{0.4pt}\\[0.3ex]
\textbf{Zuflüsse:} Lena, Yana} % 9: Anrainer + Zuflüsse

\vspace{5mm}


%----------------------------------
% CARD 11: East Siberian Sea
%----------------------------------
\SeaCard
{Ostsibirische See} % 1: German Title
{East Siberian Sea} % 2: English subtitle
{Das East Siberian Sea ist ein Randmeer des Arctic mit charakteristischen regionalen Merkmalen. Besondere Merkmale: Flach, fast ganzjährig gefroren.} % 3: Intro
{987\,000\,km²} % 4: Fläche
{58\,m} % 5: Mitt. Tiefe
{155\,m} % 6: Max Tiefe
{-2--5\,°C} % 7: Temperatur
{28--32\,‰} % 8: Salzgehalt
{\textbf{Anrainer:} Russia\\[0.3ex]
\noindent\rule{\linewidth}{0.4pt}\\[0.3ex]
\textbf{Zuflüsse:} Kolyma, Indigirka} % 9: Anrainer + Zuflüsse

\vspace{5mm}


%----------------------------------
% CARD 12: Chukchi Sea
%----------------------------------
\SeaCard
{Tschuktschensee} % 1: German Title
{Chukchi Sea} % 2: English subtitle
{Das Chukchi Sea ist ein Randmeer des Arctic mit charakteristischen regionalen Merkmalen. Besondere Merkmale: Verbindet Pazifik und Arktis.} % 3: Intro
{595\,000\,km²} % 4: Fläche
{77\,m} % 5: Mitt. Tiefe
{160\,m} % 6: Max Tiefe
{-2--12\,°C} % 7: Temperatur
{28--33\,‰} % 8: Salzgehalt
{\textbf{Anrainer:} Russia, USA (Alaska)\\[0.3ex]
\noindent\rule{\linewidth}{0.4pt}\\[0.3ex]
\textbf{Zuflüsse:} ---} % 9: Anrainer + Zuflüsse

\newpage


%----------------------------------
% CARD 13: Beaufort Sea
%----------------------------------
\SeaCard
{Beaufortsee} % 1: German Title
{Beaufort Sea} % 2: English subtitle
{Das Beaufort Sea ist ein Randmeer des Arctic mit charakteristischen regionalen Merkmalen. Besondere Merkmale: Oil/gas reserves, Eisbären.} % 3: Intro
{476\,000\,km²} % 4: Fläche
{1\,004\,m} % 5: Mitt. Tiefe
{4\,683\,m} % 6: Max Tiefe
{-2--8\,°C} % 7: Temperatur
{28--32\,‰} % 8: Salzgehalt
{\textbf{Anrainer:} Canada, USA (Alaska)\\[0.3ex]
\noindent\rule{\linewidth}{0.4pt}\\[0.3ex]
\textbf{Zuflüsse:} Mackenzie} % 9: Anrainer + Zuflüsse

\vspace{5mm}


%----------------------------------
% CARD 14: Greenland Sea
%----------------------------------
\SeaCard
{Grönlandsee} % 1: German Title
{Greenland Sea} % 2: English subtitle
{Das Greenland Sea ist ein Randmeer des Arctic mit charakteristischen regionalen Merkmalen. Besondere Merkmale: Tiefenwasserkonvektion.} % 3: Intro
{1\,205\,000\,km²} % 4: Fläche
{1\,444\,m} % 5: Mitt. Tiefe
{4\,846\,m} % 6: Max Tiefe
{-2--6\,°C} % 7: Temperatur
{30--35\,‰} % 8: Salzgehalt
{\textbf{Anrainer:} Greenland (Denmark), Iceland, Norway\\[0.3ex]
\noindent\rule{\linewidth}{0.4pt}\\[0.3ex]
\textbf{Zuflüsse:} ---} % 9: Anrainer + Zuflüsse

\vspace{5mm}


%----------------------------------
% CARD 15: Wandel Sea
%----------------------------------
\SeaCard
{Wandelsee} % 1: German Title
{Wandel Sea} % 2: English subtitle
{Das Wandel Sea ist ein Randmeer des Arctic mit charakteristischen regionalen Merkmalen. Besondere Merkmale: Extreme Eisbedingungen.} % 3: Intro
{16\,000\,km²} % 4: Fläche
{150\,m} % 5: Mitt. Tiefe
{300\,m} % 6: Max Tiefe
{-2--2\,°C} % 7: Temperatur
{30--32\,‰} % 8: Salzgehalt
{\textbf{Anrainer:} Greenland (Denmark)\\[0.3ex]
\noindent\rule{\linewidth}{0.4pt}\\[0.3ex]
\textbf{Zuflüsse:} ---} % 9: Anrainer + Zuflüsse

\vspace{5mm}


%----------------------------------
% CARD 16: Fram Strait
%----------------------------------
\SeaCard
{Framstraße} % 1: German Title
{Fram Strait} % 2: English subtitle
{Die Fram Strait ist eine Meerenge, die wichtige Meeresgebiete miteinander verbindet. Besondere Merkmale: Einzige Tiefwasserverbindung Arktis-Atlantik, 450 km breit.} % 3: Intro
{202\,500\,km²} % 4: Fläche
{1\,500\,m} % 5: Mitt. Tiefe
{5\,550\,m} % 6: Max Tiefe
{-2--5\,°C} % 7: Temperatur
{34--35\,‰} % 8: Salzgehalt
{\textbf{Anrainer:} Greenland (Denmark), Norway (Svalbard)\\[0.3ex]
\noindent\rule{\linewidth}{0.4pt}\\[0.3ex]
\textbf{Zuflüsse:} ---} % 9: Anrainer + Zuflüsse

\vspace{5mm}


%----------------------------------
% CARD 17: Norwegian Sea
%----------------------------------
\SeaCard
{Norwegische See} % 1: German Title
{Norwegian Sea} % 2: English subtitle
{Das Norwegian Sea ist ein Randmeer des North Atlantic mit charakteristischen regionalen Merkmalen. Besondere Merkmale: Golfstromeinfluss.} % 3: Intro
{1\,383\,000\,km²} % 4: Fläche
{1\,742\,m} % 5: Mitt. Tiefe
{3\,970\,m} % 6: Max Tiefe
{2--13\,°C} % 7: Temperatur
{35\,‰} % 8: Salzgehalt
{\textbf{Anrainer:} Norway, Iceland, Faroe Islands (Denmark)\\[0.3ex]
\noindent\rule{\linewidth}{0.4pt}\\[0.3ex]
\textbf{Zuflüsse:} ---} % 9: Anrainer + Zuflüsse

\vspace{5mm}


%----------------------------------
% CARD 18: North Sea
%----------------------------------
\SeaCard
{Nordsee} % 1: German Title
{North Sea} % 2: English subtitle
{Das North Sea ist ein Randmeer des North Atlantic mit charakteristischen regionalen Merkmalen. Besondere Merkmale: Wichtige Schifffahrt, Öl- und Gasindustrie.} % 3: Intro
{570\,000\,km²} % 4: Fläche
{95\,m} % 5: Mitt. Tiefe
{700\,m} % 6: Max Tiefe
{4--18\,°C} % 7: Temperatur
{32--35\,‰} % 8: Salzgehalt
{\textbf{Anrainer:} UK, Norway, Denmark, Germany, Netherlands, Belgium, France\\[0.3ex]
\noindent\rule{\linewidth}{0.4pt}\\[0.3ex]
\textbf{Zuflüsse:} Rhine, Elbe, Thames} % 9: Anrainer + Zuflüsse

\newpage


%----------------------------------
% CARD 19: Baltic Sea
%----------------------------------
\SeaCard
{Ostsee} % 1: German Title
{Baltic Sea} % 2: English subtitle
{Das Baltic Sea ist ein Randmeer des North Atlantic mit charakteristischen regionalen Merkmalen. Besondere Merkmale: Brackwasser, niedriger Salzgehalt.} % 3: Intro
{377\,000\,km²} % 4: Fläche
{55\,m} % 5: Mitt. Tiefe
{459\,m} % 6: Max Tiefe
{0--20\,°C} % 7: Temperatur
{6--20\,‰} % 8: Salzgehalt
{\textbf{Anrainer:} Sweden, Finland, Russia, Estonia, Latvia, Lithuania, Poland, Germany, Denmark\\[0.3ex]
\noindent\rule{\linewidth}{0.4pt}\\[0.3ex]
\textbf{Zuflüsse:} Neva, Vistula, Oder} % 9: Anrainer + Zuflüsse

\vspace{5mm}


%----------------------------------
% CARD 20: Mediterranean Sea
%----------------------------------
\SeaCard
{Mittelmeer} % 1: German Title
{Mediterranean Sea} % 2: English subtitle
{Das Mediterranean Sea ist ein Randmeer des North Atlantic mit charakteristischen regionalen Merkmalen. Besondere Merkmale: Hoher Salzgehalt, antike Zivilisationen.} % 3: Intro
{2\,500\,000\,km²} % 4: Fläche
{1\,500\,m} % 5: Mitt. Tiefe
{5\,267\,m} % 6: Max Tiefe
{10--30\,°C} % 7: Temperatur
{37--39\,‰} % 8: Salzgehalt
{\textbf{Anrainer:} Spain,  France,  Monaco,  Italy,  Slovenia, u.a.\\[0.3ex]
\noindent\rule{\linewidth}{0.4pt}\\[0.3ex]
\textbf{Zuflüsse:} Nile, Po, Rhône, Ebro} % 9: Anrainer + Zuflüsse

\vspace{5mm}


%----------------------------------
% CARD 21: Black Sea
%----------------------------------
\SeaCard
{Schwarzes Meer} % 1: German Title
{Black Sea} % 2: English subtitle
{Das Black Sea ist ein Randmeer des North Atlantic mit charakteristischen regionalen Merkmalen. Besondere Merkmale: Anoxische Tiefenwässer.} % 3: Intro
{436\,000\,km²} % 4: Fläche
{1\,253\,m} % 5: Mitt. Tiefe
{2\,212\,m} % 6: Max Tiefe
{6--25\,°C} % 7: Temperatur
{17--18\,‰} % 8: Salzgehalt
{\textbf{Anrainer:} Turkey, Bulgaria, Romania, Ukraine, Russia, Georgia\\[0.3ex]
\noindent\rule{\linewidth}{0.4pt}\\[0.3ex]
\textbf{Zuflüsse:} Danube, Dnieper, Don} % 9: Anrainer + Zuflüsse

\vspace{5mm}


%----------------------------------
% CARD 22: Sea of Marmara
%----------------------------------
\SeaCard
{Marmarameer} % 1: German Title
{Sea of Marmara} % 2: English subtitle
{Das Sea of Marmara ist ein Randmeer des Mediterranean mit charakteristischen regionalen Merkmalen. Besondere Merkmale: Verbindet Schwarzes Meer und Ägäis.} % 3: Intro
{11\,350\,km²} % 4: Fläche
{494\,m} % 5: Mitt. Tiefe
{1\,370\,m} % 6: Max Tiefe
{8--25\,°C} % 7: Temperatur
{22--38\,‰} % 8: Salzgehalt
{\textbf{Anrainer:} Turkey\\[0.3ex]
\noindent\rule{\linewidth}{0.4pt}\\[0.3ex]
\textbf{Zuflüsse:} ---} % 9: Anrainer + Zuflüsse

\vspace{5mm}


%----------------------------------
% CARD 23: Adriatic Sea
%----------------------------------
\SeaCard
{Adria} % 1: German Title
{Adriatic Sea} % 2: English subtitle
{Das Adriatic Sea ist ein Randmeer des Mediterranean mit charakteristischen regionalen Merkmalen. Besondere Merkmale: Tourismus, Fischerei.} % 3: Intro
{138\,600\,km²} % 4: Fläche
{252\,m} % 5: Mitt. Tiefe
{1\,233\,m} % 6: Max Tiefe
{10--28\,°C} % 7: Temperatur
{38--39\,‰} % 8: Salzgehalt
{\textbf{Anrainer:} Italy, Slovenia, Croatia, Bosnia-Herzegovina, Montenegro, Albania\\[0.3ex]
\noindent\rule{\linewidth}{0.4pt}\\[0.3ex]
\textbf{Zuflüsse:} Po} % 9: Anrainer + Zuflüsse

\vspace{5mm}


%----------------------------------
% CARD 24: Aegean Sea
%----------------------------------
\SeaCard
{Ägäis} % 1: German Title
{Aegean Sea} % 2: English subtitle
{Das Aegean Sea ist ein Randmeer des Mediterranean mit charakteristischen regionalen Merkmalen. Besondere Merkmale: Griechische Inseln, antike Geschichte.} % 3: Intro
{214\,000\,km²} % 4: Fläche
{510\,m} % 5: Mitt. Tiefe
{3\,543\,m} % 6: Max Tiefe
{14--26\,°C} % 7: Temperatur
{38--39\,‰} % 8: Salzgehalt
{\textbf{Anrainer:} Greece, Turkey\\[0.3ex]
\noindent\rule{\linewidth}{0.4pt}\\[0.3ex]
\textbf{Zuflüsse:} Maritsa} % 9: Anrainer + Zuflüsse

\newpage


%----------------------------------
% CARD 25: Caribbean Sea
%----------------------------------
\SeaCard
{Karibisches Meer} % 1: German Title
{Caribbean Sea} % 2: English subtitle
{Das Caribbean Sea ist ein Randmeer des North Atlantic mit charakteristischen regionalen Merkmalen. Besondere Merkmale: Tropisch, Hurrikane, Korallenriffe.} % 3: Intro
{2\,754\,000\,km²} % 4: Fläche
{2\,647\,m} % 5: Mitt. Tiefe
{7\,686\,m} % 6: Max Tiefe
{24--29\,°C} % 7: Temperatur
{35--36\,‰} % 8: Salzgehalt
{\textbf{Anrainer:} Mexico,  Belize,  Guatemala,  Honduras,  Nicaragua, u.a.\\[0.3ex]
\noindent\rule{\linewidth}{0.4pt}\\[0.3ex]
\textbf{Zuflüsse:} Magdalena, Coco} % 9: Anrainer + Zuflüsse

\vspace{5mm}


%----------------------------------
% CARD 26: Gulf of Mexico
%----------------------------------
\SeaCard
{Golf von Mexiko} % 1: German Title
{Gulf of Mexico} % 2: English subtitle
{Das Gulf of Mexico ist ein Randmeer des North Atlantic mit charakteristischen regionalen Merkmalen. Besondere Merkmale: Oil/gas, Hurrikane.} % 3: Intro
{1\,550\,000\,km²} % 4: Fläche
{1\,615\,m} % 5: Mitt. Tiefe
{4\,384\,m} % 6: Max Tiefe
{18--31\,°C} % 7: Temperatur
{35--36\,‰} % 8: Salzgehalt
{\textbf{Anrainer:} USA (Texas, Louisiana, Mississippi, Alabama, Florida), Mexico, Cuba\\[0.3ex]
\noindent\rule{\linewidth}{0.4pt}\\[0.3ex]
\textbf{Zuflüsse:} Mississippi, Rio Grande} % 9: Anrainer + Zuflüsse

\vspace{5mm}


%----------------------------------
% CARD 27: Labrador Sea
%----------------------------------
\SeaCard
{Labradorsee} % 1: German Title
{Labrador Sea} % 2: English subtitle
{Das Labrador Sea ist ein Randmeer des North Atlantic mit charakteristischen regionalen Merkmalen. Besondere Merkmale: Eisberge, Tiefenwasserbildung.} % 3: Intro
{841\,000\,km²} % 4: Fläche
{1\,898\,m} % 5: Mitt. Tiefe
{4\,316\,m} % 6: Max Tiefe
{-1--12\,°C} % 7: Temperatur
{31--35\,‰} % 8: Salzgehalt
{\textbf{Anrainer:} Canada, Greenland (Denmark)\\[0.3ex]
\noindent\rule{\linewidth}{0.4pt}\\[0.3ex]
\textbf{Zuflüsse:} Churchill} % 9: Anrainer + Zuflüsse

\vspace{5mm}


%----------------------------------
% CARD 28: Irminger Sea
%----------------------------------
\SeaCard
{Irmingersee} % 1: German Title
{Irminger Sea} % 2: English subtitle
{Das Irminger Sea ist ein Randmeer des North Atlantic mit charakteristischen regionalen Merkmalen. Besondere Merkmale: North Atlantic circulation.} % 3: Intro
{780\,000\,km²} % 4: Fläche
{1\,600\,m} % 5: Mitt. Tiefe
{3\,000\,m} % 6: Max Tiefe
{3--10\,°C} % 7: Temperatur
{34--35\,‰} % 8: Salzgehalt
{\textbf{Anrainer:} Iceland, Greenland (Denmark)\\[0.3ex]
\noindent\rule{\linewidth}{0.4pt}\\[0.3ex]
\textbf{Zuflüsse:} ---} % 9: Anrainer + Zuflüsse

\vspace{5mm}


%----------------------------------
% CARD 29: Baffin Bay
%----------------------------------
\SeaCard
{Baffinbucht} % 1: German Title
{Baffin Bay} % 2: English subtitle
{Das Baffin Bay ist ein Randmeer des North Atlantic mit charakteristischen regionalen Merkmalen. Besondere Merkmale: Eisberge, Eisbären.} % 3: Intro
{689\,000\,km²} % 4: Fläche
{861\,m} % 5: Mitt. Tiefe
{2\,414\,m} % 6: Max Tiefe
{-1--8\,°C} % 7: Temperatur
{30--33\,‰} % 8: Salzgehalt
{\textbf{Anrainer:} Canada, Greenland (Denmark)\\[0.3ex]
\noindent\rule{\linewidth}{0.4pt}\\[0.3ex]
\textbf{Zuflüsse:} ---} % 9: Anrainer + Zuflüsse

\vspace{5mm}


%----------------------------------
% CARD 30: Hudson Bay
%----------------------------------
\SeaCard
{Hudsonbucht} % 1: German Title
{Hudson Bay} % 2: English subtitle
{Das Hudson Bay ist ein Randmeer des North Atlantic mit charakteristischen regionalen Merkmalen. Besondere Merkmale: Flach, im Winter gefroren.} % 3: Intro
{1\,230\,000\,km²} % 4: Fläche
{128\,m} % 5: Mitt. Tiefe
{270\,m} % 6: Max Tiefe
{-2--12\,°C} % 7: Temperatur
{25--32\,‰} % 8: Salzgehalt
{\textbf{Anrainer:} Canada\\[0.3ex]
\noindent\rule{\linewidth}{0.4pt}\\[0.3ex]
\textbf{Zuflüsse:} Churchill, Nelson, Albany} % 9: Anrainer + Zuflüsse

\newpage


%----------------------------------
% CARD 31: Gulf of St. Lawrence
%----------------------------------
\SeaCard
{Sankt-Lorenz-Golf} % 1: German Title
{Gulf of St. Lawrence} % 2: English subtitle
{Das Gulf of St. Lawrence ist ein Randmeer des North Atlantic mit charakteristischen regionalen Merkmalen. Besondere Merkmale: Major estuary, Fischerei.} % 3: Intro
{236\,000\,km²} % 4: Fläche
{152\,m} % 5: Mitt. Tiefe
{572\,m} % 6: Max Tiefe
{0--20\,°C} % 7: Temperatur
{28--32\,‰} % 8: Salzgehalt
{\textbf{Anrainer:} Canada\\[0.3ex]
\noindent\rule{\linewidth}{0.4pt}\\[0.3ex]
\textbf{Zuflüsse:} St. Lawrence} % 9: Anrainer + Zuflüsse

\vspace{5mm}


%----------------------------------
% CARD 32: Irish Sea
%----------------------------------
\SeaCard
{Irische See} % 1: German Title
{Irish Sea} % 2: English subtitle
{Das Irish Sea ist ein Randmeer des North Atlantic mit charakteristischen regionalen Merkmalen. Besondere Merkmale: Flach, ferry routes.} % 3: Intro
{100\,000\,km²} % 4: Fläche
{60\,m} % 5: Mitt. Tiefe
{175\,m} % 6: Max Tiefe
{6--16\,°C} % 7: Temperatur
{32--35\,‰} % 8: Salzgehalt
{\textbf{Anrainer:} Ireland, UK\\[0.3ex]
\noindent\rule{\linewidth}{0.4pt}\\[0.3ex]
\textbf{Zuflüsse:} ---} % 9: Anrainer + Zuflüsse

\vspace{5mm}


%----------------------------------
% CARD 33: English Channel
%----------------------------------
\SeaCard
{Ärmelkanal} % 1: German Title
{English Channel} % 2: English subtitle
{Die English Channel ist eine Meerenge, die wichtige Meeresgebiete miteinander verbindet. Besondere Merkmale: Vielbefahrene Schifffahrtsroute, Tunnel.} % 3: Intro
{75\,000\,km²} % 4: Fläche
{54\,m} % 5: Mitt. Tiefe
{174\,m} % 6: Max Tiefe
{6--18\,°C} % 7: Temperatur
{34--35\,‰} % 8: Salzgehalt
{\textbf{Anrainer:} UK, France\\[0.3ex]
\noindent\rule{\linewidth}{0.4pt}\\[0.3ex]
\textbf{Zuflüsse:} Seine, Thames (estuaries)} % 9: Anrainer + Zuflüsse

\vspace{5mm}


%----------------------------------
% CARD 34: Bay of Biscay
%----------------------------------
\SeaCard
{Golf von Biskaya} % 1: German Title
{Bay of Biscay} % 2: English subtitle
{Das Bay of Biscay ist ein Randmeer des North Atlantic mit charakteristischen regionalen Merkmalen. Besondere Merkmale: Tief, raue See.} % 3: Intro
{223\,000\,km²} % 4: Fläche
{1\,744\,m} % 5: Mitt. Tiefe
{4\,735\,m} % 6: Max Tiefe
{11--20\,°C} % 7: Temperatur
{35--36\,‰} % 8: Salzgehalt
{\textbf{Anrainer:} France, Spain\\[0.3ex]
\noindent\rule{\linewidth}{0.4pt}\\[0.3ex]
\textbf{Zuflüsse:} Loire, Garonne} % 9: Anrainer + Zuflüsse

\vspace{5mm}


%----------------------------------
% CARD 35: Skagerrak
%----------------------------------
\SeaCard
{Skagerrak} % 1: German Title
{Skagerrak} % 2: English subtitle
{Die Skagerrak ist eine Meerenge, die wichtige Meeresgebiete miteinander verbindet. Besondere Merkmale: Verbindet Nordsee und Ostsee.} % 3: Intro
{32\,000\,km²} % 4: Fläche
{210\,m} % 5: Mitt. Tiefe
{700\,m} % 6: Max Tiefe
{2--17\,°C} % 7: Temperatur
{30--35\,‰} % 8: Salzgehalt
{\textbf{Anrainer:} Norway, Denmark, Sweden\\[0.3ex]
\noindent\rule{\linewidth}{0.4pt}\\[0.3ex]
\textbf{Zuflüsse:} Göta} % 9: Anrainer + Zuflüsse

\vspace{5mm}


%----------------------------------
% CARD 36: Kattegat
%----------------------------------
\SeaCard
{Kattegat} % 1: German Title
{Kattegat} % 2: English subtitle
{Das Kattegat ist ein Randmeer des North Atlantic mit charakteristischen regionalen Merkmalen. Besondere Merkmale: Übergangszone.} % 3: Intro
{30\,000\,km²} % 4: Fläche
{23\,m} % 5: Mitt. Tiefe
{109\,m} % 6: Max Tiefe
{0--20\,°C} % 7: Temperatur
{15--30\,‰} % 8: Salzgehalt
{\textbf{Anrainer:} Denmark, Sweden\\[0.3ex]
\noindent\rule{\linewidth}{0.4pt}\\[0.3ex]
\textbf{Zuflüsse:} ---} % 9: Anrainer + Zuflüsse

\newpage


%----------------------------------
% CARD 37: Strait of Gibraltar
%----------------------------------
\SeaCard
{Straße von Gibraltar} % 1: German Title
{Strait of Gibraltar} % 2: English subtitle
{Die Strait of Gibraltar ist eine Meerenge, die wichtige Meeresgebiete miteinander verbindet. Besondere Merkmale: 14.3 km an der engsten Stelle, kritischer Engpass.} % 3: Intro
{1\,050\,km²} % 4: Fläche
{300\,m} % 5: Mitt. Tiefe
{900\,m} % 6: Max Tiefe
{13--18\,°C} % 7: Temperatur
{36\,‰} % 8: Salzgehalt
{\textbf{Anrainer:} Spain, Morocco, UK (Gibraltar)\\[0.3ex]
\noindent\rule{\linewidth}{0.4pt}\\[0.3ex]
\textbf{Zuflüsse:} ---} % 9: Anrainer + Zuflüsse

\vspace{5mm}


%----------------------------------
% CARD 38: Bosporus
%----------------------------------
\SeaCard
{Bosporus} % 1: German Title
{Bosporus} % 2: English subtitle
{Die Bosporus ist eine Meerenge, die wichtige Meeresgebiete miteinander verbindet. Besondere Merkmale: 1 km an der engsten Stelle, 30 km lang, verbindet Schwarzes Meer und Mittelmeer.} % 3: Intro
{550\,km²} % 4: Fläche
{50\,m} % 5: Mitt. Tiefe
{110\,m} % 6: Max Tiefe
{6--25\,°C} % 7: Temperatur
{17--35\,‰} % 8: Salzgehalt
{\textbf{Anrainer:} Turkey\\[0.3ex]
\noindent\rule{\linewidth}{0.4pt}\\[0.3ex]
\textbf{Zuflüsse:} ---} % 9: Anrainer + Zuflüsse

\vspace{5mm}


%----------------------------------
% CARD 39: Dardanelles
%----------------------------------
\SeaCard
{Dardanellen} % 1: German Title
{Dardanelles} % 2: English subtitle
{Die Dardanelles ist eine Meerenge, die wichtige Meeresgebiete miteinander verbindet. Besondere Merkmale: 1.2 km an der engsten Stelle, 61 km lang, Teil der Türkischen Meerengen.} % 3: Intro
{1\,800\,km²} % 4: Fläche
{55\,m} % 5: Mitt. Tiefe
{103\,m} % 6: Max Tiefe
{8--24\,°C} % 7: Temperatur
{25--35\,‰} % 8: Salzgehalt
{\textbf{Anrainer:} Turkey\\[0.3ex]
\noindent\rule{\linewidth}{0.4pt}\\[0.3ex]
\textbf{Zuflüsse:} ---} % 9: Anrainer + Zuflüsse

\vspace{5mm}


%----------------------------------
% CARD 40: Gulf of Guinea
%----------------------------------
\SeaCard
{Golf von Guinea} % 1: German Title
{Gulf of Guinea} % 2: English subtitle
{Das Gulf of Guinea ist ein Randmeer des South Atlantic mit charakteristischen regionalen Merkmalen. Besondere Merkmale: Oil/gas, äquatoriale Lage.} % 3: Intro
{2\,350\,000\,km²} % 4: Fläche
{2\,996\,m} % 5: Mitt. Tiefe
{5\,469\,m} % 6: Max Tiefe
{24--28\,°C} % 7: Temperatur
{34--36\,‰} % 8: Salzgehalt
{\textbf{Anrainer:} Ghana, Togo, Benin, Nigeria, Cameroon, Equatorial Guinea, Gabon, Congo, Angola\\[0.3ex]
\noindent\rule{\linewidth}{0.4pt}\\[0.3ex]
\textbf{Zuflüsse:} Niger, Congo} % 9: Anrainer + Zuflüsse

\vspace{5mm}


%----------------------------------
% CARD 41: Weddell Sea
%----------------------------------
\SeaCard
{Weddellmeer} % 1: German Title
{Weddell Sea} % 2: English subtitle
{Das Weddell Sea ist ein Randmeer des Southern/South Atlantic mit charakteristischen regionalen Merkmalen. Besondere Merkmale: Massive Eisschelfe.} % 3: Intro
{2\,800\,000\,km²} % 4: Fläche
{3\,000\,m} % 5: Mitt. Tiefe
{6\,820\,m} % 6: Max Tiefe
{-2--2\,°C} % 7: Temperatur
{34--35\,‰} % 8: Salzgehalt
{\textbf{Anrainer:} Antarctica\\[0.3ex]
\noindent\rule{\linewidth}{0.4pt}\\[0.3ex]
\textbf{Zuflüsse:} ---} % 9: Anrainer + Zuflüsse

\vspace{5mm}


%----------------------------------
% CARD 42: Scotia Sea
%----------------------------------
\SeaCard
{Scotiasee} % 1: German Title
{Scotia Sea} % 2: English subtitle
{Das Scotia Sea ist ein Randmeer des Southern/South Atlantic mit charakteristischen regionalen Merkmalen. Besondere Merkmale: Drakestraße.} % 3: Intro
{1\,300\,000\,km²} % 4: Fläche
{3\,096\,m} % 5: Mitt. Tiefe
{6\,022\,m} % 6: Max Tiefe
{-1--6\,°C} % 7: Temperatur
{34\,‰} % 8: Salzgehalt
{\textbf{Anrainer:} Antarctica, South Georgia\\[0.3ex]
\noindent\rule{\linewidth}{0.4pt}\\[0.3ex]
\textbf{Zuflüsse:} ---} % 9: Anrainer + Zuflüsse

\newpage


%----------------------------------
% CARD 43: Bering Sea
%----------------------------------
\SeaCard
{Beringmeer} % 1: German Title
{Bering Sea} % 2: English subtitle
{Das Bering Sea ist ein Randmeer des North Pacific mit charakteristischen regionalen Merkmalen. Besondere Merkmale: Produktive Fischerei.} % 3: Intro
{2\,000\,000\,km²} % 4: Fläche
{1\,640\,m} % 5: Mitt. Tiefe
{4\,097\,m} % 6: Max Tiefe
{-2--12\,°C} % 7: Temperatur
{32--33\,‰} % 8: Salzgehalt
{\textbf{Anrainer:} Russia, USA (Alaska)\\[0.3ex]
\noindent\rule{\linewidth}{0.4pt}\\[0.3ex]
\textbf{Zuflüsse:} Yukon, Anadyr} % 9: Anrainer + Zuflüsse

\vspace{5mm}


%----------------------------------
% CARD 44: Bering Strait
%----------------------------------
\SeaCard
{Beringstraße} % 1: German Title
{Bering Strait} % 2: English subtitle
{Die Bering Strait ist eine Meerenge, die wichtige Meeresgebiete miteinander verbindet. Besondere Merkmale: 85 km breit, connects Pacific-Arctic.} % 3: Intro
{2\,300\,km²} % 4: Fläche
{45\,m} % 5: Mitt. Tiefe
{50\,m} % 6: Max Tiefe
{-2--10\,°C} % 7: Temperatur
{32--33\,‰} % 8: Salzgehalt
{\textbf{Anrainer:} Russia, USA (Alaska)\\[0.3ex]
\noindent\rule{\linewidth}{0.4pt}\\[0.3ex]
\textbf{Zuflüsse:} ---} % 9: Anrainer + Zuflüsse

\vspace{5mm}


%----------------------------------
% CARD 45: Sea of Okhotsk
%----------------------------------
\SeaCard
{Ochotskisches Meer} % 1: German Title
{Sea of Okhotsk} % 2: English subtitle
{Das Sea of Okhotsk ist ein Randmeer des North Pacific mit charakteristischen regionalen Merkmalen. Besondere Merkmale: Kältestes ostasiatisches Meer.} % 3: Intro
{1\,583\,000\,km²} % 4: Fläche
{838\,m} % 5: Mitt. Tiefe
{3\,372\,m} % 6: Max Tiefe
{-2--18\,°C} % 7: Temperatur
{32--33\,‰} % 8: Salzgehalt
{\textbf{Anrainer:} Russia, Japan\\[0.3ex]
\noindent\rule{\linewidth}{0.4pt}\\[0.3ex]
\textbf{Zuflüsse:} Amur} % 9: Anrainer + Zuflüsse

\vspace{5mm}


%----------------------------------
% CARD 46: Sea of Japan
%----------------------------------
\SeaCard
{Japanisches Meer} % 1: German Title
{Sea of Japan} % 2: English subtitle
{Das Sea of Japan ist ein Randmeer des North Pacific mit charakteristischen regionalen Merkmalen. Besondere Merkmale: Tief basin, reiche Fischgründe.} % 3: Intro
{978\,000\,km²} % 4: Fläche
{1\,667\,m} % 5: Mitt. Tiefe
{3\,742\,m} % 6: Max Tiefe
{0--25\,°C} % 7: Temperatur
{33--35\,‰} % 8: Salzgehalt
{\textbf{Anrainer:} Japan, Russia, North Korea, South Korea\\[0.3ex]
\noindent\rule{\linewidth}{0.4pt}\\[0.3ex]
\textbf{Zuflüsse:} ---} % 9: Anrainer + Zuflüsse

\vspace{5mm}


%----------------------------------
% CARD 47: Yellow Sea
%----------------------------------
\SeaCard
{Gelbes Meer} % 1: German Title
{Yellow Sea} % 2: English subtitle
{Das Yellow Sea ist ein Randmeer des North Pacific mit charakteristischen regionalen Merkmalen. Besondere Merkmale: Flach, gelbe Sedimente.} % 3: Intro
{380\,000\,km²} % 4: Fläche
{44\,m} % 5: Mitt. Tiefe
{152\,m} % 6: Max Tiefe
{0--28\,°C} % 7: Temperatur
{30--32\,‰} % 8: Salzgehalt
{\textbf{Anrainer:} China, North Korea, South Korea\\[0.3ex]
\noindent\rule{\linewidth}{0.4pt}\\[0.3ex]
\textbf{Zuflüsse:} Yellow (Huang He)} % 9: Anrainer + Zuflüsse

\vspace{5mm}


%----------------------------------
% CARD 48: East China Sea
%----------------------------------
\SeaCard
{Ostchinesisches Meer} % 1: German Title
{East China Sea} % 2: English subtitle
{Das East China Sea ist ein Randmeer des North Pacific mit charakteristischen regionalen Merkmalen. Besondere Merkmale: Kontinentalschelf.} % 3: Intro
{1\,249\,000\,km²} % 4: Fläche
{349\,m} % 5: Mitt. Tiefe
{2\,719\,m} % 6: Max Tiefe
{10--28\,°C} % 7: Temperatur
{31--34\,‰} % 8: Salzgehalt
{\textbf{Anrainer:} China, Taiwan, Japan, South Korea\\[0.3ex]
\noindent\rule{\linewidth}{0.4pt}\\[0.3ex]
\textbf{Zuflüsse:} Yangtze} % 9: Anrainer + Zuflüsse

\newpage


%----------------------------------
% CARD 49: South China Sea
%----------------------------------
\SeaCard
{Südchinesisches Meer} % 1: German Title
{South China Sea} % 2: English subtitle
{Das South China Sea ist ein Randmeer des North Pacific mit charakteristischen regionalen Merkmalen. Besondere Merkmale: Umstrittene Gebiete, Taifune.} % 3: Intro
{3\,500\,000\,km²} % 4: Fläche
{1\,652\,m} % 5: Mitt. Tiefe
{5\,559\,m} % 6: Max Tiefe
{20--30\,°C} % 7: Temperatur
{32--34\,‰} % 8: Salzgehalt
{\textbf{Anrainer:} China,  Taiwan,  Philippines,  Malaysia,  Brunei, u.a.\\[0.3ex]
\noindent\rule{\linewidth}{0.4pt}\\[0.3ex]
\textbf{Zuflüsse:} Pearl, Mekong, Red} % 9: Anrainer + Zuflüsse

\vspace{5mm}


%----------------------------------
% CARD 50: Philippine Sea
%----------------------------------
\SeaCard
{Philippinensee} % 1: German Title
{Philippine Sea} % 2: English subtitle
{Das Philippine Sea ist ein Randmeer des North Pacific mit charakteristischen regionalen Merkmalen. Besondere Merkmale: Tiefest trenches.} % 3: Intro
{5\,177\,562\,km²} % 4: Fläche
{4\,080\,m} % 5: Mitt. Tiefe
{10\,540\,m} % 6: Max Tiefe
{20--30\,°C} % 7: Temperatur
{34--35\,‰} % 8: Salzgehalt
{\textbf{Anrainer:} Philippines, Japan, Taiwan\\[0.3ex]
\noindent\rule{\linewidth}{0.4pt}\\[0.3ex]
\textbf{Zuflüsse:} Cagayan} % 9: Anrainer + Zuflüsse

\vspace{5mm}


%----------------------------------
% CARD 51: Gulf of Alaska
%----------------------------------
\SeaCard
{Golf von Alaska} % 1: German Title
{Gulf of Alaska} % 2: English subtitle
{Das Gulf of Alaska ist ein Randmeer des North Pacific mit charakteristischen regionalen Merkmalen. Besondere Merkmale: Stürmisch, reiche Fischgründe.} % 3: Intro
{1\,533\,000\,km²} % 4: Fläche
{2\,431\,m} % 5: Mitt. Tiefe
{5\,659\,m} % 6: Max Tiefe
{2--16\,°C} % 7: Temperatur
{32--33\,‰} % 8: Salzgehalt
{\textbf{Anrainer:} USA (Alaska), Canada\\[0.3ex]
\noindent\rule{\linewidth}{0.4pt}\\[0.3ex]
\textbf{Zuflüsse:} Copper, Alsek} % 9: Anrainer + Zuflüsse

\vspace{5mm}


%----------------------------------
% CARD 52: Gulf of California
%----------------------------------
\SeaCard
{Golf von Kalifornien} % 1: German Title
{Gulf of California} % 2: English subtitle
{Das Gulf of California ist ein Randmeer des North Pacific mit charakteristischen regionalen Merkmalen. Besondere Merkmale: Langer schmaler Golf.} % 3: Intro
{177\,000\,km²} % 4: Fläche
{818\,m} % 5: Mitt. Tiefe
{3\,292\,m} % 6: Max Tiefe
{16--30\,°C} % 7: Temperatur
{35\,‰} % 8: Salzgehalt
{\textbf{Anrainer:} Mexico\\[0.3ex]
\noindent\rule{\linewidth}{0.4pt}\\[0.3ex]
\textbf{Zuflüsse:} Colorado} % 9: Anrainer + Zuflüsse

\vspace{5mm}


%----------------------------------
% CARD 53: Celebes Sea
%----------------------------------
\SeaCard
{Celebessee} % 1: German Title
{Celebes Sea} % 2: English subtitle
{Das Celebes Sea ist ein Randmeer des South Pacific mit charakteristischen regionalen Merkmalen. Besondere Merkmale: Tief, warm.} % 3: Intro
{280\,000\,km²} % 4: Fläche
{4\,700\,m} % 5: Mitt. Tiefe
{6\,200\,m} % 6: Max Tiefe
{27--30\,°C} % 7: Temperatur
{34\,‰} % 8: Salzgehalt
{\textbf{Anrainer:} Philippines, Indonesia, Malaysia\\[0.3ex]
\noindent\rule{\linewidth}{0.4pt}\\[0.3ex]
\textbf{Zuflüsse:} ---} % 9: Anrainer + Zuflüsse

\vspace{5mm}


%----------------------------------
% CARD 54: Java Sea
%----------------------------------
\SeaCard
{Javasee} % 1: German Title
{Java Sea} % 2: English subtitle
{Das Java Sea ist ein Randmeer des South Pacific mit charakteristischen regionalen Merkmalen. Besondere Merkmale: Very flach.} % 3: Intro
{320\,000\,km²} % 4: Fläche
{46\,m} % 5: Mitt. Tiefe
{46\,m} % 6: Max Tiefe
{27--30\,°C} % 7: Temperatur
{32--34\,‰} % 8: Salzgehalt
{\textbf{Anrainer:} Indonesia\\[0.3ex]
\noindent\rule{\linewidth}{0.4pt}\\[0.3ex]
\textbf{Zuflüsse:} Solo} % 9: Anrainer + Zuflüsse

\newpage


%----------------------------------
% CARD 55: Banda Sea
%----------------------------------
\SeaCard
{Bandasee} % 1: German Title
{Banda Sea} % 2: English subtitle
{Das Banda Sea ist ein Randmeer des South Pacific mit charakteristischen regionalen Merkmalen. Besondere Merkmale: Tief basins, vulkanisch.} % 3: Intro
{470\,000\,km²} % 4: Fläche
{3\,060\,m} % 5: Mitt. Tiefe
{7\,440\,m} % 6: Max Tiefe
{27--30\,°C} % 7: Temperatur
{34\,‰} % 8: Salzgehalt
{\textbf{Anrainer:} Indonesia\\[0.3ex]
\noindent\rule{\linewidth}{0.4pt}\\[0.3ex]
\textbf{Zuflüsse:} ---} % 9: Anrainer + Zuflüsse

\vspace{5mm}


%----------------------------------
% CARD 56: Arafura Sea
%----------------------------------
\SeaCard
{Arafurasee} % 1: German Title
{Arafura Sea} % 2: English subtitle
{Das Arafura Sea ist ein Randmeer des South Pacific mit charakteristischen regionalen Merkmalen. Besondere Merkmale: Flach, tropisch.} % 3: Intro
{700\,000\,km²} % 4: Fläche
{80\,m} % 5: Mitt. Tiefe
{3\,680\,m} % 6: Max Tiefe
{25--29\,°C} % 7: Temperatur
{34--35\,‰} % 8: Salzgehalt
{\textbf{Anrainer:} Indonesia, Papua New Guinea, Australia\\[0.3ex]
\noindent\rule{\linewidth}{0.4pt}\\[0.3ex]
\textbf{Zuflüsse:} Fly, Digul} % 9: Anrainer + Zuflüsse

\vspace{5mm}


%----------------------------------
% CARD 57: Coral Sea
%----------------------------------
\SeaCard
{Korallenmeer} % 1: German Title
{Coral Sea} % 2: English subtitle
{Das Coral Sea ist ein Randmeer des South Pacific mit charakteristischen regionalen Merkmalen. Besondere Merkmale: Großes Barriereriff.} % 3: Intro
{4\,791\,000\,km²} % 4: Fläche
{2\,394\,m} % 5: Mitt. Tiefe
{9\,140\,m} % 6: Max Tiefe
{23--28\,°C} % 7: Temperatur
{35--36\,‰} % 8: Salzgehalt
{\textbf{Anrainer:} Australia, Papua New Guinea, Solomon Islands, Vanuatu, New Caledonia (France)\\[0.3ex]
\noindent\rule{\linewidth}{0.4pt}\\[0.3ex]
\textbf{Zuflüsse:} ---} % 9: Anrainer + Zuflüsse

\vspace{5mm}


%----------------------------------
% CARD 58: Tasman Sea
%----------------------------------
\SeaCard
{Tasmansee} % 1: German Title
{Tasman Sea} % 2: English subtitle
{Das Tasman Sea ist ein Randmeer des South Pacific mit charakteristischen regionalen Merkmalen. Besondere Merkmale: Raue See.} % 3: Intro
{2\,300\,000\,km²} % 4: Fläche
{3\,000\,m} % 5: Mitt. Tiefe
{5\,943\,m} % 6: Max Tiefe
{8--23\,°C} % 7: Temperatur
{35--35.5\,‰} % 8: Salzgehalt
{\textbf{Anrainer:} Australia, New Zealand\\[0.3ex]
\noindent\rule{\linewidth}{0.4pt}\\[0.3ex]
\textbf{Zuflüsse:} ---} % 9: Anrainer + Zuflüsse

\vspace{5mm}


%----------------------------------
% CARD 59: Solomon Sea
%----------------------------------
\SeaCard
{Salomonensee} % 1: German Title
{Solomon Sea} % 2: English subtitle
{Das Solomon Sea ist ein Randmeer des South Pacific mit charakteristischen regionalen Merkmalen. Besondere Merkmale: Tief, vulkanisch.} % 3: Intro
{720\,000\,km²} % 4: Fläche
{2\,500\,m} % 5: Mitt. Tiefe
{9\,140\,m} % 6: Max Tiefe
{26--30\,°C} % 7: Temperatur
{34--35\,‰} % 8: Salzgehalt
{\textbf{Anrainer:} Papua New Guinea, Solomon Islands\\[0.3ex]
\noindent\rule{\linewidth}{0.4pt}\\[0.3ex]
\textbf{Zuflüsse:} ---} % 9: Anrainer + Zuflüsse

\vspace{5mm}


%----------------------------------
% CARD 60: Bismarck Sea
%----------------------------------
\SeaCard
{Bismarcksee} % 1: German Title
{Bismarck Sea} % 2: English subtitle
{Das Bismarck Sea ist ein Randmeer des South Pacific mit charakteristischen regionalen Merkmalen. Besondere Merkmale: Vulkanisch.} % 3: Intro
{40\,000\,km²} % 4: Fläche
{1\,200\,m} % 5: Mitt. Tiefe
{2\,665\,m} % 6: Max Tiefe
{27--30\,°C} % 7: Temperatur
{34--35\,‰} % 8: Salzgehalt
{\textbf{Anrainer:} Papua New Guinea\\[0.3ex]
\noindent\rule{\linewidth}{0.4pt}\\[0.3ex]
\textbf{Zuflüsse:} Sepik, Ramu} % 9: Anrainer + Zuflüsse

\newpage


%----------------------------------
% CARD 61: Sulu Sea
%----------------------------------
\SeaCard
{Sulusee} % 1: German Title
{Sulu Sea} % 2: English subtitle
{Das Sulu Sea ist ein Randmeer des South Pacific mit charakteristischen regionalen Merkmalen. Besondere Merkmale: Tief basins.} % 3: Intro
{348\,000\,km²} % 4: Fläche
{1\,450\,m} % 5: Mitt. Tiefe
{5\,576\,m} % 6: Max Tiefe
{27--30\,°C} % 7: Temperatur
{34\,‰} % 8: Salzgehalt
{\textbf{Anrainer:} Philippines, Malaysia\\[0.3ex]
\noindent\rule{\linewidth}{0.4pt}\\[0.3ex]
\textbf{Zuflüsse:} ---} % 9: Anrainer + Zuflüsse

\vspace{5mm}


%----------------------------------
% CARD 62: Red Sea
%----------------------------------
\SeaCard
{Rotes Meer} % 1: German Title
{Red Sea} % 2: English subtitle
{Das Red Sea ist ein Randmeer des Indian mit charakteristischen regionalen Merkmalen. Besondere Merkmale: Höchster Salzgehalt, Korallenriffe.} % 3: Intro
{438\,000\,km²} % 4: Fläche
{490\,m} % 5: Mitt. Tiefe
{3\,040\,m} % 6: Max Tiefe
{20--30\,°C} % 7: Temperatur
{40--41\,‰} % 8: Salzgehalt
{\textbf{Anrainer:} Egypt, Sudan, Eritrea, Djibouti, Yemen, Saudi Arabia\\[0.3ex]
\noindent\rule{\linewidth}{0.4pt}\\[0.3ex]
\textbf{Zuflüsse:} ---} % 9: Anrainer + Zuflüsse

\vspace{5mm}


%----------------------------------
% CARD 63: Bab-el-Mandeb
%----------------------------------
\SeaCard
{Bab al-Mandab} % 1: German Title
{Bab-el-Mandeb} % 2: English subtitle
{Die Bab-el-Mandeb ist eine Meerenge, die wichtige Meeresgebiete miteinander verbindet. Besondere Merkmale: 26 km an der engsten Stelle, critical oil chokepoint.} % 3: Intro
{5\,000\,km²} % 4: Fläche
{150\,m} % 5: Mitt. Tiefe
{310\,m} % 6: Max Tiefe
{22--28\,°C} % 7: Temperatur
{36\,‰} % 8: Salzgehalt
{\textbf{Anrainer:} Yemen, Djibouti, Eritrea\\[0.3ex]
\noindent\rule{\linewidth}{0.4pt}\\[0.3ex]
\textbf{Zuflüsse:} ---} % 9: Anrainer + Zuflüsse

\vspace{5mm}


%----------------------------------
% CARD 64: Persian Gulf
%----------------------------------
\SeaCard
{Persischer Golf} % 1: German Title
{Persian Gulf} % 2: English subtitle
{Das Persian Gulf ist ein Randmeer des Indian mit charakteristischen regionalen Merkmalen. Besondere Merkmale: Große Ölreserven.} % 3: Intro
{251\,000\,km²} % 4: Fläche
{50\,m} % 5: Mitt. Tiefe
{90\,m} % 6: Max Tiefe
{15--35\,°C} % 7: Temperatur
{37--41\,‰} % 8: Salzgehalt
{\textbf{Anrainer:} Iran, Iraq, Kuwait, Saudi Arabia, Bahrain, Qatar, UAE, Oman\\[0.3ex]
\noindent\rule{\linewidth}{0.4pt}\\[0.3ex]
\textbf{Zuflüsse:} Tigris-Euphrates (Shatt al-Arab)} % 9: Anrainer + Zuflüsse

\vspace{5mm}


%----------------------------------
% CARD 65: Strait of Hormuz
%----------------------------------
\SeaCard
{Straße von Hormuz} % 1: German Title
{Strait of Hormuz} % 2: English subtitle
{Die Strait of Hormuz ist eine Meerenge, die wichtige Meeresgebiete miteinander verbindet. Besondere Merkmale: 35-60 km breit, critical oil chokepoint.} % 3: Intro
{3\,900\,km²} % 4: Fläche
{75\,m} % 5: Mitt. Tiefe
{100\,m} % 6: Max Tiefe
{18--33\,°C} % 7: Temperatur
{37--40\,‰} % 8: Salzgehalt
{\textbf{Anrainer:} Iran, Oman, UAE\\[0.3ex]
\noindent\rule{\linewidth}{0.4pt}\\[0.3ex]
\textbf{Zuflüsse:} ---} % 9: Anrainer + Zuflüsse

\vspace{5mm}


%----------------------------------
% CARD 66: Arabian Sea
%----------------------------------
\SeaCard
{Arabisches Meer} % 1: German Title
{Arabian Sea} % 2: English subtitle
{Das Arabian Sea ist ein Randmeer des Indian mit charakteristischen regionalen Merkmalen. Besondere Merkmale: Monsoon influence.} % 3: Intro
{3\,862\,000\,km²} % 4: Fläche
{2\,734\,m} % 5: Mitt. Tiefe
{4\,652\,m} % 6: Max Tiefe
{22--28\,°C} % 7: Temperatur
{36--37\,‰} % 8: Salzgehalt
{\textbf{Anrainer:} India, Pakistan, Iran, Oman, Yemen, Maldives, Somalia\\[0.3ex]
\noindent\rule{\linewidth}{0.4pt}\\[0.3ex]
\textbf{Zuflüsse:} Indus} % 9: Anrainer + Zuflüsse

\newpage


%----------------------------------
% CARD 67: Bay of Bengal
%----------------------------------
\SeaCard
{Golf von Bengalen} % 1: German Title
{Bay of Bengal} % 2: English subtitle
{Das Bay of Bengal ist ein Randmeer des Indian mit charakteristischen regionalen Merkmalen. Besondere Merkmale: Zyklone, große Deltas.} % 3: Intro
{2\,172\,000\,km²} % 4: Fläche
{2\,600\,m} % 5: Mitt. Tiefe
{4\,694\,m} % 6: Max Tiefe
{25--30\,°C} % 7: Temperatur
{30--34\,‰} % 8: Salzgehalt
{\textbf{Anrainer:} India, Bangladesh, Myanmar, Thailand, Sri Lanka\\[0.3ex]
\noindent\rule{\linewidth}{0.4pt}\\[0.3ex]
\textbf{Zuflüsse:} Ganges-Brahmaputra, Godavari, Mahanadi, Irrawaddy} % 9: Anrainer + Zuflüsse

\vspace{5mm}


%----------------------------------
% CARD 68: Andaman Sea
%----------------------------------
\SeaCard
{Andamanensee} % 1: German Title
{Andaman Sea} % 2: English subtitle
{Das Andaman Sea ist ein Randmeer des Indian mit charakteristischen regionalen Merkmalen. Besondere Merkmale: 2004 Tsunami.} % 3: Intro
{797\,700\,km²} % 4: Fläche
{1\,096\,m} % 5: Mitt. Tiefe
{4\,180\,m} % 6: Max Tiefe
{26--30\,°C} % 7: Temperatur
{30--33\,‰} % 8: Salzgehalt
{\textbf{Anrainer:} Myanmar, Thailand, Malaysia, Indonesia, India\\[0.3ex]
\noindent\rule{\linewidth}{0.4pt}\\[0.3ex]
\textbf{Zuflüsse:} Irrawaddy, Salween} % 9: Anrainer + Zuflüsse

\vspace{5mm}


%----------------------------------
% CARD 69: Strait of Malacca
%----------------------------------
\SeaCard
{Straße von Malakka} % 1: German Title
{Strait of Malacca} % 2: English subtitle
{Die Strait of Malacca ist eine Meerenge, die wichtige Meeresgebiete miteinander verbindet. Besondere Merkmale: 2.8 km an der engsten Stelle point, meistbefahrene Schifffahrtsroute.} % 3: Intro
{80\,000\,km²} % 4: Fläche
{15\,m} % 5: Mitt. Tiefe
{25\,m} % 6: Max Tiefe
{26--30\,°C} % 7: Temperatur
{30--34\,‰} % 8: Salzgehalt
{\textbf{Anrainer:} Malaysia, Indonesia, Singapore\\[0.3ex]
\noindent\rule{\linewidth}{0.4pt}\\[0.3ex]
\textbf{Zuflüsse:} ---} % 9: Anrainer + Zuflüsse

\vspace{5mm}


%----------------------------------
% CARD 70: Timor Sea
%----------------------------------
\SeaCard
{Timorsee} % 1: German Title
{Timor Sea} % 2: English subtitle
{Das Timor Sea ist ein Randmeer des Indian mit charakteristischen regionalen Merkmalen. Besondere Merkmale: Oil/gas.} % 3: Intro
{610\,000\,km²} % 4: Fläche
{435\,m} % 5: Mitt. Tiefe
{3\,310\,m} % 6: Max Tiefe
{25--30\,°C} % 7: Temperatur
{34--35\,‰} % 8: Salzgehalt
{\textbf{Anrainer:} Indonesia, Timor-Leste, Australia\\[0.3ex]
\noindent\rule{\linewidth}{0.4pt}\\[0.3ex]
\textbf{Zuflüsse:} ---} % 9: Anrainer + Zuflüsse

\vspace{5mm}


%----------------------------------
% CARD 71: Mozambique Channel
%----------------------------------
\SeaCard
{Straße von Mosambik} % 1: German Title
{Mozambique Channel} % 2: English subtitle
{Das Mozambique Channel ist ein Randmeer des Indian mit charakteristischen regionalen Merkmalen. Besondere Merkmale: Schifffahrtsroute.} % 3: Intro
{700\,000\,km²} % 4: Fläche
{1\,600\,m} % 5: Mitt. Tiefe
{3\,292\,m} % 6: Max Tiefe
{24--28\,°C} % 7: Temperatur
{35\,‰} % 8: Salzgehalt
{\textbf{Anrainer:} Mozambique, Madagascar\\[0.3ex]
\noindent\rule{\linewidth}{0.4pt}\\[0.3ex]
\textbf{Zuflüsse:} Zambezi} % 9: Anrainer + Zuflüsse

\vspace{5mm}


%----------------------------------
% CARD 72: Suez Canal
%----------------------------------
\SeaCard
{Sueskanal} % 1: German Title
{Suez Canal} % 2: English subtitle
{Der Suez Canal ist ein künstlicher Wasserweg von großer strategischer Bedeutung. Besondere Merkmale: 193 km lang, verbindet Mittelmeer und Rotes Meer.} % 3: Intro
{195\,km²} % 4: Fläche
{13\,m} % 5: Mitt. Tiefe
{24\,m} % 6: Max Tiefe
{18--30\,°C} % 7: Temperatur
{40--42\,‰} % 8: Salzgehalt
{\textbf{Anrainer:} Egypt\\[0.3ex]
\noindent\rule{\linewidth}{0.4pt}\\[0.3ex]
\textbf{Zuflüsse:} ---} % 9: Anrainer + Zuflüsse

\newpage


%----------------------------------
% CARD 73: Panama Canal
%----------------------------------
\SeaCard
{Panamakanal} % 1: German Title
{Panama Canal} % 2: English subtitle
{Der Panama Canal ist ein künstlicher Wasserweg von großer strategischer Bedeutung. Besondere Merkmale: 82 km lang, verbindet Atlantik und Pazifik.} % 3: Intro
{425\,km²} % 4: Fläche
{13\,m} % 5: Mitt. Tiefe
{25\,m} % 6: Max Tiefe
{24--32\,°C} % 7: Temperatur
{0--35\,‰} % 8: Salzgehalt
{\textbf{Anrainer:} Panama\\[0.3ex]
\noindent\rule{\linewidth}{0.4pt}\\[0.3ex]
\textbf{Zuflüsse:} Chagres (for locks)} % 9: Anrainer + Zuflüsse

\vspace{5mm}


%----------------------------------
% CARD 74: Caspian Sea
%----------------------------------
\SeaCard
{Kaspisches Meer} % 1: German Title
{Caspian Sea} % 2: English subtitle
{Der Caspian Sea ist ein Salz- bzw. Brackwassersee mit einzigartigen Eigenschaften. Besondere Merkmale: World's largest lake, einschließlich Garabogazköl.} % 3: Intro
{389\,000\,km²} % 4: Fläche
{211\,m} % 5: Mitt. Tiefe
{1\,025\,m} % 6: Max Tiefe
{0--27\,°C} % 7: Temperatur
{12\,‰} % 8: Salzgehalt
{\textbf{Anrainer:} Iran, Kazakhstan, Turkmenistan, Azerbaijan, Russia\\[0.3ex]
\noindent\rule{\linewidth}{0.4pt}\\[0.3ex]
\textbf{Zuflüsse:} Volga, Ural, Kura} % 9: Anrainer + Zuflüsse

\vspace{5mm}


%----------------------------------
% CARD 75: Lake Superior
%----------------------------------
\SeaCard
{Oberer See} % 1: German Title
{Lake Superior} % 2: English subtitle
{Der Lake Superior ist ein Süßwassersee mit bedeutendem ökologischen Wert. Besondere Merkmale: Größter Süßwassersee nach Fläche.} % 3: Intro
{82\,100\,km²} % 4: Fläche
{147\,m} % 5: Mitt. Tiefe
{406\,m} % 6: Max Tiefe
{0--13\,°C} % 7: Temperatur
{0\,‰} % 8: Salzgehalt
{\textbf{Anrainer:} Canada, USA (Michigan, Wisconsin, Minnesota)\\[0.3ex]
\noindent\rule{\linewidth}{0.4pt}\\[0.3ex]
\textbf{Zuflüsse:} Nipigon, St. Louis, Pigeon} % 9: Anrainer + Zuflüsse

\vspace{5mm}


%----------------------------------
% CARD 76: Lake Victoria
%----------------------------------
\SeaCard
{Victoriasee} % 1: German Title
{Lake Victoria} % 2: English subtitle
{Der Lake Victoria ist ein Süßwassersee mit bedeutendem ökologischen Wert. Besondere Merkmale: Größter See Afrikas.} % 3: Intro
{59\,940\,km²} % 4: Fläche
{40\,m} % 5: Mitt. Tiefe
{81\,m} % 6: Max Tiefe
{20--27\,°C} % 7: Temperatur
{0\,‰} % 8: Salzgehalt
{\textbf{Anrainer:} Tanzania, Uganda, Kenya\\[0.3ex]
\noindent\rule{\linewidth}{0.4pt}\\[0.3ex]
\textbf{Zuflüsse:} Kagera} % 9: Anrainer + Zuflüsse

\vspace{5mm}


%----------------------------------
% CARD 77: Lake Huron
%----------------------------------
\SeaCard
{Huronsee} % 1: German Title
{Lake Huron} % 2: English subtitle
{Der Lake Huron ist ein Süßwassersee mit bedeutendem ökologischen Wert. Besondere Merkmale: Enthält Manitoulin Island.} % 3: Intro
{59\,590\,km²} % 4: Fläche
{59\,m} % 5: Mitt. Tiefe
{229\,m} % 6: Max Tiefe
{0--22\,°C} % 7: Temperatur
{0\,‰} % 8: Salzgehalt
{\textbf{Anrainer:} Canada, USA (Michigan)\\[0.3ex]
\noindent\rule{\linewidth}{0.4pt}\\[0.3ex]
\textbf{Zuflüsse:} ---} % 9: Anrainer + Zuflüsse

\vspace{5mm}


%----------------------------------
% CARD 78: Lake Michigan
%----------------------------------
\SeaCard
{Michigansee} % 1: German Title
{Lake Michigan} % 2: English subtitle
{Der Lake Michigan ist ein Süßwassersee mit bedeutendem ökologischen Wert. Besondere Merkmale: Einziger Großer See vollständig in den USA.} % 3: Intro
{57\,800\,km²} % 4: Fläche
{85\,m} % 5: Mitt. Tiefe
{282\,m} % 6: Max Tiefe
{0--24\,°C} % 7: Temperatur
{0\,‰} % 8: Salzgehalt
{\textbf{Anrainer:} USA (Michigan, Wisconsin, Illinois, Indiana)\\[0.3ex]
\noindent\rule{\linewidth}{0.4pt}\\[0.3ex]
\textbf{Zuflüsse:} Fox-Wolf, Grand, Kalamazoo} % 9: Anrainer + Zuflüsse

\newpage


%----------------------------------
% CARD 79: Lake Tanganyika
%----------------------------------
\SeaCard
{Tanganjikasee} % 1: German Title
{Lake Tanganyika} % 2: English subtitle
{Der Lake Tanganyika ist ein Süßwassersee mit bedeutendem ökologischen Wert. Besondere Merkmale: Längster Süßwassersee, zweittiefster.} % 3: Intro
{32\,900\,km²} % 4: Fläche
{570\,m} % 5: Mitt. Tiefe
{1\,470\,m} % 6: Max Tiefe
{24--29\,°C} % 7: Temperatur
{0\,‰} % 8: Salzgehalt
{\textbf{Anrainer:} Tanzania, DRC, Burundi, Zambia\\[0.3ex]
\noindent\rule{\linewidth}{0.4pt}\\[0.3ex]
\textbf{Zuflüsse:} Ruzizi, Malagarasi} % 9: Anrainer + Zuflüsse

\vspace{5mm}


%----------------------------------
% CARD 80: Lake Baikal
%----------------------------------
\SeaCard
{Baikalsee} % 1: German Title
{Lake Baikal} % 2: English subtitle
{Der Lake Baikal ist ein Süßwassersee mit bedeutendem ökologischen Wert. Besondere Merkmale: Tiefest and oldest lake.} % 3: Intro
{31\,722\,km²} % 4: Fläche
{744\,m} % 5: Mitt. Tiefe
{1\,642\,m} % 6: Max Tiefe
{3--10\,°C} % 7: Temperatur
{0\,‰} % 8: Salzgehalt
{\textbf{Anrainer:} Russia\\[0.3ex]
\noindent\rule{\linewidth}{0.4pt}\\[0.3ex]
\textbf{Zuflüsse:} Selenga, Barguzin, Upper Angara} % 9: Anrainer + Zuflüsse

\vspace{5mm}


%----------------------------------
% CARD 81: Great Bear Lake
%----------------------------------
\SeaCard
{Großer Bärensee} % 1: German Title
{Great Bear Lake} % 2: English subtitle
{Der Great Bear Lake ist ein Süßwassersee mit bedeutendem ökologischen Wert. Besondere Merkmale: Größter See vollständig in Kanada.} % 3: Intro
{31\,153\,km²} % 4: Fläche
{72\,m} % 5: Mitt. Tiefe
{446\,m} % 6: Max Tiefe
{0--18\,°C} % 7: Temperatur
{0\,‰} % 8: Salzgehalt
{\textbf{Anrainer:} Canada\\[0.3ex]
\noindent\rule{\linewidth}{0.4pt}\\[0.3ex]
\textbf{Zuflüsse:} Dease, Whitefish} % 9: Anrainer + Zuflüsse

\vspace{5mm}


%----------------------------------
% CARD 82: Lake Malawi (Nyasa)
%----------------------------------
\SeaCard
{Malawisee} % 1: German Title
{Lake Malawi (Nyasa)} % 2: English subtitle
{Der Lake Malawi (Nyasa) ist ein Süßwassersee mit bedeutendem ökologischen Wert. Besondere Merkmale: Meisten Fischarten aller Seen.} % 3: Intro
{29\,600\,km²} % 4: Fläche
{292\,m} % 5: Mitt. Tiefe
{706\,m} % 6: Max Tiefe
{22--28\,°C} % 7: Temperatur
{0\,‰} % 8: Salzgehalt
{\textbf{Anrainer:} Malawi, Mozambique, Tanzania\\[0.3ex]
\noindent\rule{\linewidth}{0.4pt}\\[0.3ex]
\textbf{Zuflüsse:} Ruhuhu} % 9: Anrainer + Zuflüsse

\vspace{5mm}


%----------------------------------
% CARD 83: Great Slave Lake
%----------------------------------
\SeaCard
{Großer Sklavensee} % 1: German Title
{Great Slave Lake} % 2: English subtitle
{Der Great Slave Lake ist ein Süßwassersee mit bedeutendem ökologischen Wert. Besondere Merkmale: Tiefest lake in North America.} % 3: Intro
{27\,200\,km²} % 4: Fläche
{41\,m} % 5: Mitt. Tiefe
{614\,m} % 6: Max Tiefe
{0--18\,°C} % 7: Temperatur
{0\,‰} % 8: Salzgehalt
{\textbf{Anrainer:} Canada\\[0.3ex]
\noindent\rule{\linewidth}{0.4pt}\\[0.3ex]
\textbf{Zuflüsse:} Slave, Hay, Taltson} % 9: Anrainer + Zuflüsse

\vspace{5mm}


%----------------------------------
% CARD 84: Lake Erie
%----------------------------------
\SeaCard
{Eriesee} % 1: German Title
{Lake Erie} % 2: English subtitle
{Der Lake Erie ist ein Süßwassersee mit bedeutendem ökologischen Wert. Besondere Merkmale: Kleinstes Volumen der Großen Seen.} % 3: Intro
{25\,667\,km²} % 4: Fläche
{19\,m} % 5: Mitt. Tiefe
{64\,m} % 6: Max Tiefe
{0--24\,°C} % 7: Temperatur
{0\,‰} % 8: Salzgehalt
{\textbf{Anrainer:} Canada, USA (Michigan, Ohio, Pennsylvania, New York)\\[0.3ex]
\noindent\rule{\linewidth}{0.4pt}\\[0.3ex]
\textbf{Zuflüsse:} Detroit, Maumee, Grand} % 9: Anrainer + Zuflüsse

\newpage


%----------------------------------
% CARD 85: Lake Winnipeg
%----------------------------------
\SeaCard
{Winnipegsee} % 1: German Title
{Lake Winnipeg} % 2: English subtitle
{Der Lake Winnipeg ist ein Süßwassersee mit bedeutendem ökologischen Wert. Besondere Merkmale: Sehr großes Einzugsgebiet.} % 3: Intro
{24\,514\,km²} % 4: Fläche
{12\,m} % 5: Mitt. Tiefe
{36\,m} % 6: Max Tiefe
{0--22\,°C} % 7: Temperatur
{0\,‰} % 8: Salzgehalt
{\textbf{Anrainer:} Canada\\[0.3ex]
\noindent\rule{\linewidth}{0.4pt}\\[0.3ex]
\textbf{Zuflüsse:} Red, Saskatchewan, Winnipeg} % 9: Anrainer + Zuflüsse

\vspace{5mm}


%----------------------------------
% CARD 86: Lake Ontario
%----------------------------------
\SeaCard
{Ontariosee} % 1: German Title
{Lake Ontario} % 2: English subtitle
{Der Lake Ontario ist ein Süßwassersee mit bedeutendem ökologischen Wert. Besondere Merkmale: Am tiefsten liegender der Großen Seen.} % 3: Intro
{18\,970\,km²} % 4: Fläche
{86\,m} % 5: Mitt. Tiefe
{244\,m} % 6: Max Tiefe
{0--20\,°C} % 7: Temperatur
{0\,‰} % 8: Salzgehalt
{\textbf{Anrainer:} Canada, USA (New York)\\[0.3ex]
\noindent\rule{\linewidth}{0.4pt}\\[0.3ex]
\textbf{Zuflüsse:} Niagara, Genesee} % 9: Anrainer + Zuflüsse

\vspace{5mm}


%----------------------------------
% CARD 87: Lake Ladoga
%----------------------------------
\SeaCard
{Ladogasee} % 1: German Title
{Lake Ladoga} % 2: English subtitle
{Der Lake Ladoga ist ein Süßwassersee mit bedeutendem ökologischen Wert. Besondere Merkmale: Größter See Europas.} % 3: Intro
{17\,700\,km²} % 4: Fläche
{51\,m} % 5: Mitt. Tiefe
{260\,m} % 6: Max Tiefe
{0--20\,°C} % 7: Temperatur
{0\,‰} % 8: Salzgehalt
{\textbf{Anrainer:} Russia\\[0.3ex]
\noindent\rule{\linewidth}{0.4pt}\\[0.3ex]
\textbf{Zuflüsse:} Svir, Volkhov, Neva} % 9: Anrainer + Zuflüsse

\vspace{5mm}


%----------------------------------
% CARD 88: Lake Balkhash
%----------------------------------
\SeaCard
{Balchaschsee} % 1: German Title
{Lake Balkhash} % 2: English subtitle
{Der Lake Balkhash ist ein Salz- bzw. Brackwassersee mit einzigartigen Eigenschaften. Besondere Merkmale: Westhälfte süß, Osthälfte salzig.} % 3: Intro
{16\,400\,km²} % 4: Fläche
{5\,m} % 5: Mitt. Tiefe
{26\,m} % 6: Max Tiefe
{0--28\,°C} % 7: Temperatur
{3\,‰} % 8: Salzgehalt
{\textbf{Anrainer:} Kazakhstan\\[0.3ex]
\noindent\rule{\linewidth}{0.4pt}\\[0.3ex]
\textbf{Zuflüsse:} Ili} % 9: Anrainer + Zuflüsse

\vspace{5mm}


%----------------------------------
% CARD 89: Lake Vostok
%----------------------------------
\SeaCard
{Wostoksee} % 1: German Title
{Lake Vostok} % 2: English subtitle
{Der Lake Vostok ist ein subglazialer See, der unter einer dicken Eisschicht verborgen liegt. Besondere Merkmale: Unter 4 km Eis begraben.} % 3: Intro
{12\,500\,km²} % 4: Fläche
{430\,m} % 5: Mitt. Tiefe
{900\,m} % 6: Max Tiefe
{-3---2\,°C} % 7: Temperatur
{0\,‰} % 8: Salzgehalt
{\textbf{Anrainer:} Antarctica\\[0.3ex]
\noindent\rule{\linewidth}{0.4pt}\\[0.3ex]
\textbf{Zuflüsse:} ---} % 9: Anrainer + Zuflüsse

\vspace{5mm}


%----------------------------------
% CARD 90: Lake Onega
%----------------------------------
\SeaCard
{Onegasee} % 1: German Title
{Lake Onega} % 2: English subtitle
{Der Lake Onega ist ein Süßwassersee mit bedeutendem ökologischen Wert. Besondere Merkmale: Zweitgrößter See Europas.} % 3: Intro
{9\,700\,km²} % 4: Fläche
{30\,m} % 5: Mitt. Tiefe
{127\,m} % 6: Max Tiefe
{0--20\,°C} % 7: Temperatur
{0\,‰} % 8: Salzgehalt
{\textbf{Anrainer:} Russia\\[0.3ex]
\noindent\rule{\linewidth}{0.4pt}\\[0.3ex]
\textbf{Zuflüsse:} Suna, Vodla} % 9: Anrainer + Zuflüsse

\newpage


%----------------------------------
% CARD 91: Lake Titicaca
%----------------------------------
\SeaCard
{Titicacasee} % 1: German Title
{Lake Titicaca} % 2: English subtitle
{Der Lake Titicaca ist ein Süßwassersee mit bedeutendem ökologischen Wert. Besondere Merkmale: Höchster schiffbarer See (3,812 m).} % 3: Intro
{8\,372\,km²} % 4: Fläche
{107\,m} % 5: Mitt. Tiefe
{281\,m} % 6: Max Tiefe
{10--16\,°C} % 7: Temperatur
{0\,‰} % 8: Salzgehalt
{\textbf{Anrainer:} Peru, Bolivia\\[0.3ex]
\noindent\rule{\linewidth}{0.4pt}\\[0.3ex]
\textbf{Zuflüsse:} Ramis, Ilave} % 9: Anrainer + Zuflüsse

\vspace{5mm}


%----------------------------------
% CARD 92: Lake Nicaragua
%----------------------------------
\SeaCard
{Nicaraguasee} % 1: German Title
{Lake Nicaragua} % 2: English subtitle
{Der Lake Nicaragua ist ein Süßwassersee mit bedeutendem ökologischen Wert. Besondere Merkmale: Größter See Mittelamerikas.} % 3: Intro
{8\,264\,km²} % 4: Fläche
{13\,m} % 5: Mitt. Tiefe
{26\,m} % 6: Max Tiefe
{25--29\,°C} % 7: Temperatur
{0\,‰} % 8: Salzgehalt
{\textbf{Anrainer:} Nicaragua\\[0.3ex]
\noindent\rule{\linewidth}{0.4pt}\\[0.3ex]
\textbf{Zuflüsse:} Tipitapa, San Juan} % 9: Anrainer + Zuflüsse

\vspace{5mm}


%----------------------------------
% CARD 93: Lake Athabasca
%----------------------------------
\SeaCard
{Athabascasee} % 1: German Title
{Lake Athabasca} % 2: English subtitle
{Der Lake Athabasca ist ein Süßwassersee mit bedeutendem ökologischen Wert. Besondere Merkmale: Überrest des glazialen Lake McConnell.} % 3: Intro
{7\,850\,km²} % 4: Fläche
{20\,m} % 5: Mitt. Tiefe
{124\,m} % 6: Max Tiefe
{0--20\,°C} % 7: Temperatur
{0\,‰} % 8: Salzgehalt
{\textbf{Anrainer:} Canada\\[0.3ex]
\noindent\rule{\linewidth}{0.4pt}\\[0.3ex]
\textbf{Zuflüsse:} Athabasca, Fond du Lac} % 9: Anrainer + Zuflüsse

\vspace{5mm}


%----------------------------------
% CARD 94: Reindeer Lake
%----------------------------------
\SeaCard
{Reindeer Lake} % 1: German Title
{Reindeer Lake} % 2: English subtitle
{Der Reindeer Lake ist ein Süßwassersee mit bedeutendem ökologischen Wert. Besondere Merkmale: Meteoreinschlagstelle in tiefsten Bereichen.} % 3: Intro
{6\,650\,km²} % 4: Fläche
{53\,m} % 5: Mitt. Tiefe
{219\,m} % 6: Max Tiefe
{0--18\,°C} % 7: Temperatur
{0\,‰} % 8: Salzgehalt
{\textbf{Anrainer:} Canada\\[0.3ex]
\noindent\rule{\linewidth}{0.4pt}\\[0.3ex]
\textbf{Zuflüsse:} Reindeer} % 9: Anrainer + Zuflüsse

\vspace{5mm}


%----------------------------------
% CARD 95: Lake Turkana
%----------------------------------
\SeaCard
{Turkana-See} % 1: German Title
{Lake Turkana} % 2: English subtitle
{Der Lake Turkana ist ein Salz- bzw. Brackwassersee mit einzigartigen Eigenschaften. Besondere Merkmale: Größter Wüsten- und alkalischer See.} % 3: Intro
{6\,405\,km²} % 4: Fläche
{30\,m} % 5: Mitt. Tiefe
{109\,m} % 6: Max Tiefe
{28--31\,°C} % 7: Temperatur
{2.5\,‰} % 8: Salzgehalt
{\textbf{Anrainer:} Kenya, Ethiopia\\[0.3ex]
\noindent\rule{\linewidth}{0.4pt}\\[0.3ex]
\textbf{Zuflüsse:} Omo, Turkwel} % 9: Anrainer + Zuflüsse

\vspace{5mm}


%----------------------------------
% CARD 96: Issyk-Kul
%----------------------------------
\SeaCard
{Yssykköl} % 1: German Title
{Issyk-Kul} % 2: English subtitle
{Der Issyk-Kul ist ein Salz- bzw. Brackwassersee mit einzigartigen Eigenschaften. Besondere Merkmale: Zweitgrößter Gebirgssee, friert nie.} % 3: Intro
{6\,236\,km²} % 4: Fläche
{278\,m} % 5: Mitt. Tiefe
{668\,m} % 6: Max Tiefe
{4--18\,°C} % 7: Temperatur
{6\,‰} % 8: Salzgehalt
{\textbf{Anrainer:} Kyrgyzstan\\[0.3ex]
\noindent\rule{\linewidth}{0.4pt}\\[0.3ex]
\textbf{Zuflüsse:} Jyrgalang, Tyup} % 9: Anrainer + Zuflüsse

\newpage


%----------------------------------
% CARD 97: Lake Vänern
%----------------------------------
\SeaCard
{Vänern} % 1: German Title
{Lake Vänern} % 2: English subtitle
{Der Lake Vänern ist ein Süßwassersee mit bedeutendem ökologischen Wert. Besondere Merkmale: Größter See Europasan Union.} % 3: Intro
{5\,650\,km²} % 4: Fläche
{27\,m} % 5: Mitt. Tiefe
{106\,m} % 6: Max Tiefe
{2--18\,°C} % 7: Temperatur
{0\,‰} % 8: Salzgehalt
{\textbf{Anrainer:} Sweden\\[0.3ex]
\noindent\rule{\linewidth}{0.4pt}\\[0.3ex]
\textbf{Zuflüsse:} Klarälven, Gullspång} % 9: Anrainer + Zuflüsse

\vspace{5mm}


%----------------------------------
% CARD 98: Dead Sea
%----------------------------------
\SeaCard
{Totes Meer} % 1: German Title
{Dead Sea} % 2: English subtitle
{Der Dead Sea ist ein Salz- bzw. Brackwassersee mit einzigartigen Eigenschaften. Besondere Merkmale: Tiefster Punkt der Erde (-430m), 9,6-mal salziger als Ozean.} % 3: Intro
{605\,km²} % 4: Fläche
{120\,m} % 5: Mitt. Tiefe
{304\,m} % 6: Max Tiefe
{20--32\,°C} % 7: Temperatur
{342\,‰} % 8: Salzgehalt
{\textbf{Anrainer:} Israel, Jordan, Palestine (West Bank)\\[0.3ex]
\noindent\rule{\linewidth}{0.4pt}\\[0.3ex]
\textbf{Zuflüsse:} Jordan River} % 9: Anrainer + Zuflüsse

\vspace{5mm}


%----------------------------------
% CARD 99: Lake Rukwa
%----------------------------------
\SeaCard
{Rukwasee} % 1: German Title
{Lake Rukwa} % 2: English subtitle
{Der Lake Rukwa ist ein Salz- bzw. Brackwassersee mit einzigartigen Eigenschaften. Besondere Merkmale: Tanzania's 3rd largest lake, saisonale Schwankungen.} % 3: Intro
{5\,615\,km²} % 4: Fläche
{11\,m} % 5: Mitt. Tiefe
{22\,m} % 6: Max Tiefe
{20--28\,°C} % 7: Temperatur
{Variable\,‰} % 8: Salzgehalt
{\textbf{Anrainer:} Tanzania\\[0.3ex]
\noindent\rule{\linewidth}{0.4pt}\\[0.3ex]
\textbf{Zuflüsse:} Songwe} % 9: Anrainer + Zuflüsse

\vspace{5mm}


%----------------------------------
% CARD 100: Lake Albert
%----------------------------------
\SeaCard
{Albertsee} % 1: German Title
{Lake Albert} % 2: English subtitle
{Der Lake Albert ist ein Süßwassersee mit bedeutendem ökologischen Wert. Besondere Merkmale: Teil des Ostafrikanischen Grabensystems.} % 3: Intro
{5\,590\,km²} % 4: Fläche
{25\,m} % 5: Mitt. Tiefe
{51\,m} % 6: Max Tiefe
{26--30\,°C} % 7: Temperatur
{0\,‰} % 8: Salzgehalt
{\textbf{Anrainer:} Uganda, DRC\\[0.3ex]
\noindent\rule{\linewidth}{0.4pt}\\[0.3ex]
\textbf{Zuflüsse:} Victoria Nile, Semliki} % 9: Anrainer + Zuflüsse

\vspace{5mm}


%----------------------------------
% CARD 101: Nettilling Lake
%----------------------------------
\SeaCard
{Nettilling Lake} % 1: German Title
{Nettilling Lake} % 2: English subtitle
{Der Nettilling Lake ist ein Süßwassersee mit bedeutendem ökologischen Wert. Besondere Merkmale: Größter See auf einer Insel (Baffin Island).} % 3: Intro
{5\,542\,km²} % 4: Fläche
{53\,m} % 5: Mitt. Tiefe
{132\,m} % 6: Max Tiefe
{0--10\,°C} % 7: Temperatur
{0\,‰} % 8: Salzgehalt
{\textbf{Anrainer:} Canada\\[0.3ex]
\noindent\rule{\linewidth}{0.4pt}\\[0.3ex]
\textbf{Zuflüsse:} ---} % 9: Anrainer + Zuflüsse

\vspace{5mm}


%----------------------------------
% CARD 102: Lake Winnipegosis
%----------------------------------
\SeaCard
{Winnipegosis-See} % 1: German Title
{Lake Winnipegosis} % 2: English subtitle
{Der Lake Winnipegosis ist ein Süßwassersee mit bedeutendem ökologischen Wert. Besondere Merkmale: Zweitgrößter See in Manitoba.} % 3: Intro
{5\,370\,km²} % 4: Fläche
{5\,m} % 5: Mitt. Tiefe
{12\,m} % 6: Max Tiefe
{0--22\,°C} % 7: Temperatur
{0\,‰} % 8: Salzgehalt
{\textbf{Anrainer:} Canada\\[0.3ex]
\noindent\rule{\linewidth}{0.4pt}\\[0.3ex]
\textbf{Zuflüsse:} Mossy, Red Deer} % 9: Anrainer + Zuflüsse

\newpage


%----------------------------------
% CARD 103: Lake Mweru
%----------------------------------
\SeaCard
{Mwerusee} % 1: German Title
{Lake Mweru} % 2: English subtitle
{Der Lake Mweru ist ein Süßwassersee mit bedeutendem ökologischen Wert. Besondere Merkmale: Zweitgrößter See im Kongo-Einzugsgebiet.} % 3: Intro
{5\,120\,km²} % 4: Fläche
{7\,m} % 5: Mitt. Tiefe
{27\,m} % 6: Max Tiefe
{24--28\,°C} % 7: Temperatur
{0\,‰} % 8: Salzgehalt
{\textbf{Anrainer:} Zambia, DRC\\[0.3ex]
\noindent\rule{\linewidth}{0.4pt}\\[0.3ex]
\textbf{Zuflüsse:} Luapula} % 9: Anrainer + Zuflüsse

\vspace{5mm}


%----------------------------------
% CARD 104: Lake Nipigon
%----------------------------------
\SeaCard
{Nipigon-See} % 1: German Title
{Lake Nipigon} % 2: English subtitle
{Der Lake Nipigon ist ein Süßwassersee mit bedeutendem ökologischen Wert. Besondere Merkmale: Größter See vollständig in Ontario, Teil des Großen-Seen-Beckens.} % 3: Intro
{4\,848\,km²} % 4: Fläche
{52\,m} % 5: Mitt. Tiefe
{165\,m} % 6: Max Tiefe
{0--20\,°C} % 7: Temperatur
{0\,‰} % 8: Salzgehalt
{\textbf{Anrainer:} Canada\\[0.3ex]
\noindent\rule{\linewidth}{0.4pt}\\[0.3ex]
\textbf{Zuflüsse:} Ombabika, Gull} % 9: Anrainer + Zuflüsse

\vspace{5mm}


%----------------------------------
% CARD 105: Lake Manitoba
%----------------------------------
\SeaCard
{Manitobasee} % 1: German Title
{Lake Manitoba} % 2: English subtitle
{Der Lake Manitoba ist ein Salz- bzw. Brackwassersee mit einzigartigen Eigenschaften. Besondere Merkmale: Überrest des glazialen Lake Agassiz, very flach.} % 3: Intro
{4\,706\,km²} % 4: Fläche
{3\,m} % 5: Mitt. Tiefe
{7\,m} % 6: Max Tiefe
{0--22\,°C} % 7: Temperatur
{3.5\,‰} % 8: Salzgehalt
{\textbf{Anrainer:} Canada\\[0.3ex]
\noindent\rule{\linewidth}{0.4pt}\\[0.3ex]
\textbf{Zuflüsse:} Whitemud, Waterhen} % 9: Anrainer + Zuflüsse

\vspace{5mm}


%----------------------------------
% CARD 106: Lake Taymyr
%----------------------------------
\SeaCard
{Taimyrsee} % 1: German Title
{Lake Taymyr} % 2: English subtitle
{Der Lake Taymyr ist ein Süßwassersee mit bedeutendem ökologischen Wert. Besondere Merkmale: Größter See vollständig innerhalb des Polarkreises.} % 3: Intro
{4\,560\,km²} % 4: Fläche
{2\,m} % 5: Mitt. Tiefe
{26\,m} % 6: Max Tiefe
{-20--15\,°C} % 7: Temperatur
{0\,‰} % 8: Salzgehalt
{\textbf{Anrainer:} Russia\\[0.3ex]
\noindent\rule{\linewidth}{0.4pt}\\[0.3ex]
\textbf{Zuflüsse:} Upper Taymyr} % 9: Anrainer + Zuflüsse

\vspace{5mm}


%----------------------------------
% CARD 107: Qinghai Lake
%----------------------------------
\SeaCard
{Qinghai-See} % 1: German Title
{Qinghai Lake} % 2: English subtitle
{Der Qinghai Lake ist ein Salz- bzw. Brackwassersee mit einzigartigen Eigenschaften. Besondere Merkmale: Größter See Chinas, abflussloses Becken.} % 3: Intro
{4\,489\,km²} % 4: Fläche
{21\,m} % 5: Mitt. Tiefe
{32\,m} % 6: Max Tiefe
{-5--20\,°C} % 7: Temperatur
{14\,‰} % 8: Salzgehalt
{\textbf{Anrainer:} China\\[0.3ex]
\noindent\rule{\linewidth}{0.4pt}\\[0.3ex]
\textbf{Zuflüsse:} Buha, Shaliu} % 9: Anrainer + Zuflüsse

\vspace{5mm}


%----------------------------------
% CARD 108: Lake Saimaa
%----------------------------------
\SeaCard
{Saimaa} % 1: German Title
{Lake Saimaa} % 2: English subtitle
{Der Lake Saimaa ist ein Süßwassersee mit bedeutendem ökologischen Wert. Besondere Merkmale: Größter See Finnlands, größter in den nordischen Ländern.} % 3: Intro
{4\,380\,km²} % 4: Fläche
{17\,m} % 5: Mitt. Tiefe
{82\,m} % 6: Max Tiefe
{0--20\,°C} % 7: Temperatur
{0\,‰} % 8: Salzgehalt
{\textbf{Anrainer:} Finland\\[0.3ex]
\noindent\rule{\linewidth}{0.4pt}\\[0.3ex]
\textbf{Zuflüsse:} Vuoksi} % 9: Anrainer + Zuflüsse

\newpage


%----------------------------------
% CARD 109: Lake of the Woods
%----------------------------------
\SeaCard
{Lake of the Woods} % 1: German Title
{Lake of the Woods} % 2: English subtitle
{Der Lake of the Woods ist ein Süßwassersee mit bedeutendem ökologischen Wert. Besondere Merkmale: Über 14.500 Inseln, 105,000 km Küstenlinie.} % 3: Intro
{4\,350\,km²} % 4: Fläche
{9\,m} % 5: Mitt. Tiefe
{64\,m} % 6: Max Tiefe
{0--22\,°C} % 7: Temperatur
{0\,‰} % 8: Salzgehalt
{\textbf{Anrainer:} Canada, USA (Minnesota)\\[0.3ex]
\noindent\rule{\linewidth}{0.4pt}\\[0.3ex]
\textbf{Zuflüsse:} Rainy River} % 9: Anrainer + Zuflüsse

\vspace{5mm}


%----------------------------------
% CARD 110: Lake Khanka
%----------------------------------
\SeaCard
{Chankasee} % 1: German Title
{Lake Khanka} % 2: English subtitle
{Der Lake Khanka ist ein Süßwassersee mit bedeutendem ökologischen Wert. Besondere Merkmale: Entwässert durch den Amur ins Japanische Meer.} % 3: Intro
{4\,190\,km²} % 4: Fläche
{4\,m} % 5: Mitt. Tiefe
{10\,m} % 6: Max Tiefe
{0--25\,°C} % 7: Temperatur
{0\,‰} % 8: Salzgehalt
{\textbf{Anrainer:} Russia, China\\[0.3ex]
\noindent\rule{\linewidth}{0.4pt}\\[0.3ex]
\textbf{Zuflüsse:} Songacha, Melgunovka} % 9: Anrainer + Zuflüsse

\end{document}
