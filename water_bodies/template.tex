\documentclass[a4paper]{article}
\usepackage[margin=12mm]{geometry}
\usepackage{graphicx}
\usepackage{array}
\usepackage{lmodern}
\usepackage[T1]{fontenc}
\usepackage{microtype}
\setlength{\parindent}{0pt}

% Card dimensions (59 x 91 mm)
\newlength{\cardW}
\newlength{\cardH}
\setlength{\cardW}{59mm}
\setlength{\cardH}{91mm}

% Command to produce a card frame
\newcommand{\SeaCard}[9]{%
\fbox{\begin{minipage}[t][\dimexpr\cardH-2\fboxsep-2\fboxrule\relax][t]{\dimexpr\cardW-2\fboxsep-2\fboxrule\relax}
\raggedright
\setlength{\baselineskip}{8pt}
% Header with decorative lines
\noindent\hfill{\tiny\rule{8mm}{0.4pt} \textbf{MEERESGEBIETE} \rule{8mm}{0.4pt}}\hfill\mbox{}\\[0.5ex]
\noindent\rule{\linewidth}{1pt}\\[0.5ex]
% Title
{\Large\textbf{#1}}\\[0.3ex]
\noindent\rule{30mm}{0.4pt}\\[1ex]
% Illustration placeholder
\noindent\fbox{\begin{minipage}[c][18mm]{\dimexpr\linewidth-2\fboxsep-2\fboxrule\relax}
\centering
\vfill
{\scriptsize [Illustration]}
\vfill
\end{minipage}}\\[1ex]
% Intro text
{\scriptsize #2}\\[1ex]
\noindent\rule{\linewidth}{0.4pt}\\[0.3ex]
% Statistics table
{\scriptsize
\begin{tabular}{@{}l@{\hspace{1mm}}|@{\hspace{1mm}}l@{}}
\textbf{Fläche:} #3 & \textbf{Volumen:} #4 \\
\end{tabular}
}\\[-0.5ex]
\noindent\rule{\linewidth}{0.4pt}\\[0.3ex]
{\scriptsize
\begin{tabular}{@{}l@{\hspace{1mm}}|@{\hspace{1mm}}l@{}}
\textbf{Mittlere Tiefe:} #6 & \textbf{Max. Tiefe:} #5 \\
\end{tabular}
}\\[-0.5ex]
\noindent\rule{\linewidth}{0.4pt}\\[0.3ex]
{\scriptsize
\begin{tabular}{@{}l@{\hspace{1mm}}|@{\hspace{1mm}}l@{}}
\textbf{Temp.:} #7 & \textbf{Salzgehalt:} #8 \\
\end{tabular}
}\\[-0.5ex]
\noindent\rule{\linewidth}{0.4pt}\\[0.3ex]
% Anrainer
{\scriptsize\textbf{Anrainer:} #9}\\[0.3ex]
\end{minipage}}%
}

\begin{document}
\pagestyle{empty}

%----------------------------------
% NORDSEE CARD
%----------------------------------
\SeaCard
{Nordsee} % 1: Title
{Die Nordsee ist ein flaches Schelfmeer mit ausgeprägten Gezeiten – bei Ebbe fallen weite Wattflächen trocken. Starke Strömungen, Stürme und die Vermischung von Salz- und Süßwasser prägen dieses dynamische Ökosystem, das Millionen Zugvögeln als Rastplatz dient.} % 2: Intro
{570\,000\,km²} % 3: Fläche
{54\,000\,km³} % 4: Volumen
{700\,m} % 5: Max Tiefe
{95\,m} % 6: Mittlere Tiefe
{4--18\,°C} % 7: Temperatur
{32--35\,\%} % 8: Salzgehalt
{UK, NO, DK, DE, NL, BE, FR} % 9: Anrainer

\vspace{15mm}

%----------------------------------
% OSTSEE CARD
%----------------------------------
\SeaCard
{Ostsee} % 1: Title
{Die Ostsee ist ein brackiges Binnenmeer mit eingeschränktem Wasseraustausch zum Atlantik. Geringe Salinität, schwache Gezeiten und ein schichtungsanfälliges Wassersystem prägen die regionalspezifische Fauna und Flora. Das Meer dient Zugvögeln als Rastplatz und verbindet zahlreiche Staaten.} % 2: Intro
{377\,000\,km²} % 3: Fläche
{21\,700\,km³} % 4: Volumen
{459\,m} % 5: Max Tiefe
{55\,m} % 6: Mittlere Tiefe
{2--17\,°C} % 7: Temperatur
{1--20\,\%} % 8: Salzgehalt
{DE, DK, SE, FI, RU, EE, LV, LT, PL} % 9: Anrainer

\end{document}
